\documentclass{article}
\usepackage[utf8]{inputenc}
\usepackage{todonotes}
\usepackage[colorlinks=true, allcolors=blue]{hyperref}
\usepackage{algorithm}
\usepackage{algorithmic}
\usepackage{multirow}
\usepackage{amsmath}




\title{rapport}
\author{jjycavailles }
\date{May 2019}

%\usepackage{natbib}

%\usepackage[authoryear]{natbib}
%\usepackage{biblatex}
\usepackage[round]{natbib}
%\usepackage{natbib}

%\bibliographystyle{plainnat}
%\usepackage[round]{natbib}

%\usepackage{cite}
%\setcitestyle{authoryear, open={((},close={))}

%\bibliography{references.bib}
%\bibliography{intro.bib}
%\addbibresource{intro.bib}
%\bibliographystyle{apalike}
%\usepackage[style=authoryear]{biblatex}
%\setcitestyle{authoryear,open={((},close={))}}

%\bibliographystyle{apalike}


\usepackage{graphicx}


%\usepackage{hyperref}
\usepackage[colorlinks=true, allcolors=blue]{hyperref}





\begin{document}

\begin{titlepage}

\begin{center}
  \includegraphics[width = 25mm]{LogoInsa.png} \hfill
  \includegraphics[width = 30mm]{logo_cnrs.jpg}
\end{center}

%\title{\textbf{Rapport de stage} \\ Ingenieur en mathematique \\ \textbf{ etude de la classification en regimes de temps et de leurs impacts pour des utilisations metiers.} }
%\author{Jerome Cavailles \\ $4^{ieme}$ annee, Genie mathematique et modelisation \\ INSA Toulouse}
%\date{28/06/2018 - 07/09/2018}




\vspace*{1cm}

\begin{center}
\rule{\linewidth}{0.7mm} \\
[0.4cm]
\textbf{ \Huge Internship report} \\
[0.2cm]
\large \emph{Engineer in mathematics} \\ 
[0.6cm]
\textbf{ \huge Destabilizing effects of controlling ecosystem behavior} \\
[0.4cm]
03/01/2019 - 08/30/2019 \\
[0.4cm]
\rule{\linewidth}{0.7mm}
\end{center}

\vspace*{0.5cm}

\begin{center}
\textbf{\Large{Jerome Cavailles}} \footnote{\url{jcavaill@etud.insa-toulouse.fr}} \\ [0.3cm] $5$ years, Mathematical engineering and modeling %\\ INSA Toulouse
\end{center}

%^{\text{ieme}}

%\maketitle

%\vspace*{2.5cm}
\vspace*{1.9cm}

\begin{flushleft}
Supervisor: Yuval Zelnik \footnote{\url{yuval.zelnik@sete.cnrs.fr}} \hfill
Tutor: Marie-Hélène Vignal \footnote{\url{marie-helene.vignal@math.univ-toulouse.fr}} \\  
Michel Loreau \footnote{\url{michel.loreau@sete.cnrs.fr}} \hfill
Charles Dossal \footnote{\url{dossal@insa-toulouse.fr}} \\
Laboratory: CNRS-Moulis \footnote{2, route du CNRS - 09200 Moulis, France, \url{http://www.cbtm-moulis.com}} \hfill 
University: INSA Toulouse \footnote{135, Avenue de Rangueil 31077 Toulouse Cedex 4, \url{http://www.insa-toulouse.fr}} \\
\hfill Paul Sabatier \footnote{118 route de Narbonne, 31062 Toulouse Cedex 9, \url{http://www.univ-tlse3.fr/}}
\end{flushleft}

%\paragraph{}
%\begin{tabbing}
%\hspace{2cm}\=\hspace{6cm}\=\hspace{2cm}\=\kill
%Supervisor \> Yuval Zelnik \footnote{\url{yuval.zelnik@sete.cnrs.fr}} \>
%Tutor \> Marie-Hélène Vignal \footnote{\url{marie-helene.vignal@math.univ-toulouse.fr}} \\  
%\> Michel Loreau \footnote{\url{michel.loreau@sete.cnrs.fr}} 
%\> \> Charles Dossal \footnote{\url{dossal@insa-toulouse.fr}} \\
%Laboratory \> CNRS-Moulis \footnote{2, route du CNRS - 09200 Moulis, France, \url{http://www.cbtm-moulis.com}} \> University \> INSA Toulouse\footnote{135, Avenue de Rangueil 31077 Toulouse Cedex 4} \\
%\>\>\> Paul Sabatier \footnote{118 route de Narbonne, 31062 TOULOUSE CEDEX 9 \url{http://www.univ-tlse3.fr/}}
%\end{tabbing}

\end{titlepage}







\newpage
%\addto\captionsfrench{\def\contentsname{}} % pour supprimer le "table des matieres en haut"
\paragraph{}
%\section*{Contents}
\addcontentsline{toc}{section}{Contents}
\tableofcontents



\newpage
%\section*{Liste des figures, des tables et des algorithmes}
%\addcontentsline{toc}{section}{Liste des figures, des tables et des algorithmes}
%\paragraph{}
\addcontentsline{toc}{section}{List of figures}
\listoffigures



%\newpage

%\listoftables

%\listofalgorithms




\newpage
\section*{Acknowledgment}
\addcontentsline{toc}{section}{Acknowledgment}


\paragraph{}
Throughout the writing of this dissertation I have received a great deal of support and assistance.

\paragraph{}
I would first like to thank my memory advisor Yuval Zelnik , for his patience, motivation and enthusiasm. You provided me with the tools that I needed to choose the right direction and successfully complete my dissertation. Your willingness to give your time so generously has been very much appreciated. 

\paragraph{}
I would also like to acknowledge Michel Loreau for their encouragement and insightful comments. Your expertise was invaluable in the formulating of the research topic and methodology. You supported me greatly and were always willing to help me.

\paragraph{}
I want to thank all the staff of the moulis station for your excellent cooperation and for the opportunities you give me.

\paragraph{}
I would also like to thank my jury members, professor  Marie-Hélène Vignal and professor Charles Dossal, for dedicated time for this memories.

\paragraph{}
Finally, I must express my profound gratitude to my parents for providing me with unfailing support and encouragement throughout my years of study. 

\paragraph{}
Jérôme Cavaillès



\newpage
%\todo{Choice of this internship ?}
\section*{Introduction}
\addcontentsline{toc}{section}{Introduction}

\subsection*{Station presentation}
\addcontentsline{toc}{subsection}{Station presentation}

\paragraph{}
The CNRS\footnote{\url{http://www.cnrs.fr/en/cnrs}}, the Scientific Research National Center (in french, Centre National de la Recherche Scientifique) was created on the 19th of October, 1939. It is a world renowned research institution, ranked second by nature index \footnote{\url{https://www.natureindex.com/institution-outputs/generate/All/global/All/n_article}}. It has approximately 33,000 researchers working in 1,144 laboratories throughout France and abroad, with a budget approximately 3 billion euros. 

The CNRS is currently headed by Antoine Petit (President and CEO), and its laboratories are divided into two categories: proper units (UPRs) and mixed units (UMRs), the latter being managed in association with other French institutions (higher education establishment or another research institutions). In addition, there are 36 international Joint Units (UMI) of collaborations around the world. The CNRS conducts research in all disciplines of basic research (Ecology and environment, Humanities and social sciences, Engineering and systems, Mathematics, Physics, Information sciences, etc.).

One of these mixed research units (UMR 5321) is the Station for Theoretical and Experimental Ecology\footnote{\url{https://sete-moulis-cnrs.fr}} (SETE), located in Moulis\footnote{\url{http://www.communes.com/midi-pyrenees/ariege/moulis_09200/}} (Ariege, France). It was originally founded in 1948 by researchers Jeannel and Vandel, due to its vicinity to many caves, with the aim of the station to use the underground cave systems in order to study the formation and physical properties of karstic systems as well as systematics and adaptations in hypogeaic organisms. More recently, under the direction of Jean Clobert\footnote{\url{http://www.sete.cnrs.fr/spip.php?article26}}, the station transitioned to perform more general research about ecology.

The research station is now directed by Michel Loreau\footnote{\url{http://www.cbtm-moulis.com/m-171-michel-loreau.html}}, and has a staff of 60 persons working in it, divided into three teams.
The evolutionary ecology team studies empirically how biodiversity is generated and how species adapt to new contexts \footnote{\url{https://sete-moulis-cnrs.fr/en/research/evol}}. A second team is eco-evolutionary dynamics in changing landscapes, which aims at understanding reciprocal eco-evolutionary dynamics in landscapes modified by human activities \footnote{\url{https://sete-moulis-cnrs.fr/en/research/eedyl}}. The third team is the centre for biodiversity theory and modelling (CBTM) \footnote{\url{http://www.cbtm-moulis.com}}, which aims to unify theories of biodiversity changes and of their consequences, in order to address the major challenges of the present biodiversity crisis.

Several unique experimental platforms are located within the station, which includes a unique laboratory built inside the cave system, a $750m^2$ greenhouse which has been recently constructed, and a $520m^2$ aviary with an automatic system for data capture using video and sensors. The station also has equipment for molecular biology, cell biology, physiology, for surgery and also to breeding of invertebrates, fish, amphibians, and reptiles. 

However, the most unique facility of the station is the metatron\footnote{\url{https://themetatron.weebly.com/}}, which is a network of 48 interconnected cells. Each cell has a surface of $100m^2$ and a height of 2 m, which reproduces a small ecosystem, with different vegetation (50 species per cage) and invertebrates (40 families per cage). Cells can be linked to study the dispersal of the species from one cell to others. By controlling the temperature and other environmental conditions, it is possible to investigate the consequences of the global change. For example, a gradient of temperature can be applied to monitor the distribution of different populations. In the last year an additional facility has been built, with similar goals and setup to the metatron, but for aquatic environments: the aquatron. Similarly to the metatron, it is a network of interconnected cells. Each of the 144 cells is a basin of approximately $2m^3$, and together they are used to study the impact of climate change on aquatic species.




\begin{figure}[h]
\begin{center}
\includegraphics[width=6.cm]{metatron_0.jpg}
\includegraphics[width=6.cm]{aquatron.png}
\end{center}
\caption{\label{fig:temp}Left: Metatron, right: Aquatron}
\end{figure}

%% YZ: There's also the aqua-tron, that has recently been finished. You can talk to several people in Jose's team about it, if you're interested.


\paragraph{}
The centre for biodiversity theory and modelling (CBTM) \footnote{\url{http://www.cbtm-moulis.com}} aims to unify theories of biodiversity changes and of their consequences, in order to address the major challenges of the present biodiversity crisis. Lead by Jose M. Montoya\footnote{\url{http://www.cbtm-moulis.com/m-224-jose-m--montoya.html}}, the research ranges from phylogenetics to human interactions. Indeed, the ambition of the team is to develop a theoretical framework to a general biodiversity science in order to deal with the present biodiversity crisis\footnote{\url{https://www.ipbes.net/news/Media-Release-Global-Assessment-Fr}}.

In practice, the team focuses on several axes: generation of biodiversity and ecosystem services, human nature interactions, habitat fragmentation and stability of ecological systems.

The first axis focuses on understanding how biodiversity change will affect ecosystem services. For example, how the loss of a species can affect crop production. Given the relationships between ecological and elocutionary processes, the team integrates the role of eco-evolutionary dynamics in the responses of environmental changes. The same work is also done for spatial structure, dynamics and functioning of the ecosystem, all of which are interconnected \citep{bastazini_loss_2017, bideault_temperature_2019, galiana_geographical_2019}.

An additional axis is of human-nature interactions, which is studied in both directions: the human impact on biodiversity (e.g. habitat loss, fragmentation, global warming) which represent a threat of at least one in six species during this century, but also the feedback of biodiversity loss on human society. The aim is to study the long term sustainability of coupled social-ecological systems. Another objective is to better understand how biodiversity changes affect the services of the ecosystem, in particular crop production and biological control in agricultural landscapes \citep{cazalis_we_2018, lafuite_sustainable_2018, montoya_trade-offs_2018, montoya_tradeoffs_2019}.

% Habitat fragmentation and spatial dynamics of biodiversity
A particular case of human impact, landscape fragmentation which occurs at multiple spatial scales, is itself a focus of research: both the habitat loss and its effect on the spatial dynamics of the ecosystem are studies, with a special attention to metacommunities dynamics \citep{goncalves_habitat_2018, jacobi_operationalizing_2018}.

% Biodiversity and stability of ecological systems
Finally, a major focus of the team is studying the stability of ecological systems\footnote{\url{http://www.cbtm-moulis.com/m-214-biostases.html}}. Stability is an important feature of the ecosystem, and is a notion that is used in all previous research axes mentioned. The main research focus is to understand and quantify stability both in time and in space \citep{wang_stability_2017, zelnik_impact_2018}. However, since different stability measures are used in theory and experimental ecology, the team has tried to bridge the gap between these different measures and thus unify the notion of the stability \citep{arnold_examination_nodate}. Different aspect of stability are studied: the link between the diversity of a species community and the stability of the community \citep{vallina2017phytoplankton}, as well as the stability of meta-ecosystems \citep{arnoldi_particularity_2016, lurgi_effects_2016, wang_biodiversity_2016}.
Here as well, the sustainability of coupled social ecological systems are studied, in particular the role of human behavior in preventing the possible collapse of this systems.
Connecting all these studies, a mathematical framework is built by exploring the different notions of stability in order to link them \citep{arnoldi2016unifying, donohue_navigating_2016} and to know how to predict a critical changes by using experimental measures such as temporal variability \citep{arnoldi2016resilience, haegeman_resilience_2016, wang_invariability-area_2017}. 
% citation jusqu'a 2016 inclus

\newpage


\subsection*{Context}
\addcontentsline{toc}{subsection}{Context}

\subsubsection*{Ecosystem stability} % monitoring ?
\addcontentsline{toc}{subsubsection}{Ecosystem stability}

\paragraph{}
Ecosystems around the world are facing unprecedented disturbances due to increasing human intervention \citep{oosthoek_humanity_2005}. It is therefore important to understand their dynamics in the context of human perturbations, in order to both predict the future state of the ecosystem and to better manage it.
Given the complexity that ecosystem dynamics can take, driven by the interactions between the different interconnected elements that constitute the system, it is highly useful to study ecosystems with an integrative approach, considering both the internal dynamics as well as the external disturbances that it undergoes.
%Thus, it is unavoidable to consider integrative study. 

%\todo[color =  red]{New paragraph about disturbances/perturbations}
\paragraph{} % paragraph about disturbances/perturbations
According to \cite{rykiel_towards_1985}, a disturbance is defined  as events that can cause significant changes to the ecosystem \citep{white1985natural, rykiel_towards_1985}, by either directly affecting the current state of the ecosystem, or its conditions.. The disturbance are also complex in the sense that their different characteristics (e.g. strength, frequency, periodicity, pulse, press \citep{bender1984perturbation}) could imply a diversity of perturbations (collapse, short-long term response \citep{arnoldi2018ecosystems}). The tolerance of these perturbations are studied under the general concept of stability.

\label{stability_litterature}
Ecologists are often interested in estimating stability far from equilibrium (as opposed to the classical physics approach, which focuses on studying stability near the equilibrium). Therefore, ecology needs to develop tools to deal with understanding and predicting stability in complex dynamical systems. 
% resilience
One such concept, resilience, (the maximum disturbance strength that an ecosystem could withstand without changing structure \citep{holling_resilience_1973}) is traditionally used in theoretical studies, but is not the most relevant %\citep{arnoldi2016resilience}
\citep{gunderson_ecological_2000, neubert_alternatives_1997}.

Other less used stability measures have been established to quantify the health of an ecosystem and follow its development over time, and are used to set accurate goals for the future planning management \citep {donohue_navigating_2016, mayer_strengths_2008}. One major difficulty is that the term stability is used for many different meanings. One study has identified 163 definitions of 70 different stability concepts \citep{grimm_babel_1997}.

However, according to the same study, all these can be collapsed to only 6 pertinent concepts (constancy,  resilience,  persistence,  resistance,  elasticity and domain of attraction). Nonetheless, even if it is possible to reduce the number of the stability notions, several of them need to be used in order to consider the various aspects of stability, so as not to lose information on the behavior of the ecosystem \citep{derissen_relationship_2011}. There is thus a compromise between using too few measures, which will not capture all the relevant information, and using too many measures, which will not be practicable and may still not catch all the relevant information for decision makers \citep{hillebrand_decomposing_2018, donohue_dimensionality_2013}.
%Indeed, different stability measures have been defined to monitor different aspects of the system's response to perturbations. 
%If this different measures appear to be strongly related, it is not always true \citep{donohue2013dimensionality}.

\label{def_detection}
These different concepts could notably used to avoid collapse. Some study have shown the relevance to use this theory to predict the risk to collapse \citep{carr_modeling_2012, dai_generic_2012, dai_slower_2013}.
It exists a literature which focuses on detection of critical transitions, i.e. shifts in the system's state that occur abruptly and cannot easily be reversed. It is highly useful to be able to anticipate such abrupt changes, in order to prevent them, or at the very least to decrease their effect. One of the most used indicators of such critical change is \textit{variability} \citep{brock_variance_2006, carpenter2006rising, scheffer2015generic, dakos_robustness_2012, biggs_turning_2009}, while many others indicators exist as well \citep{scheffer_generic_2015,dakos_methods_2012}. Most of these methods are based on at least one of the following phenomena: critical slowing down and flickering.

The theory of \textit{critical slowing down} focuses on the slowdown of dynamics due to the proximity with a tipping point \citep{dakos_critical_2014, dakos_slowing_nodate, scheffer_anticipating_2012}. Many studies use this principle to develop early warning signals in different field such as psychology \citep{van_de_leemput_critical_2014}, climate change \citep{lenton_early_2012}, ecology \citep{chisholm_critical_2009, gandhi_critical_1998},  engineering \citep{ren_early_2015}, and finance \citep{diks_critical_2018}. The basic idea is that when a system approaches a tipping point, the basin of attraction flattens, and if perturbations are of the same amplitude, then the variance changes. Fig. \ref{fig:csd} shows how variability increases when the basin of attraction changes.

\begin{figure}[h]
\begin{center}
\includegraphics[width=12cm]{slowing_down.png}
\end{center}
\caption{\label{fig:temp}Heuristic illustration of critical slowing down could not be used directly. from \cite{lenton_early_2012}.
\label{fig:csd}
}
\end{figure}

Flickering is based on the idea that close to a tipping point, the dynamics will change from one basin of attraction to another one. This affects the distribution of the state of the system \citep{carr_modeling_2012, wang_flickering_2012, dakos_flickering_2013, scheffer_anticipating_2012}. Fig. \ref{fig:flick} gives an illustration of this phenomena.


\begin{figure}[h]
\begin{center}
\includegraphics[width=10cm]{flickering.png}
\end{center}
\caption{\label{fig:temp}Flickering between alternative states in highly stochastic systems \citep{scheffer_anticipating_2012}}
\label{fig:flick}
\end{figure}


%\paragraph{ecosystem management \\}
\paragraph{}
Studying the stability of ecosystem is of high interest for ecosystem management \citep{mumby_ecological_2014}. They are useful when predicting the consequences of disturbances, in particular for anthropocentric ones. In the past decades, the field of ecosystem management has grown rapidly \citep{grumbine_reflections_1997} in response to the various modern disturbances, in order to sustain the integrity of ecosystem ,including its structure, composition and function \citep{jensen1994overview}. 

A major obstacle is to define measurable goal in order to have clear and trackable progress of the ecosystem state \citep{slocombe_forum:_1998}. Even if it is not realistic to know all the exact processes operating within the ecosystem, it is still often possible to understand the dominant behavior, which could be sufficient for ecosystem management \citep{mori_ecosystem_2011, slocombe_forum:_1998, stanley_ecosystem_1995}.
%\citep{mori2011ecosystem, slocombe1998defining, stanley1995ecosystem}

\subsubsection*{Forest fire}
\addcontentsline{toc}{subsubsection}{Forest fire}

\paragraph{}
A notable example of an ecosystem where both different stability measures and management decisions are highly relevant is a forest that undergoes repeated wildfires. We thus focus on forest fire management, as the dynamics of both forest and fire are well established. Nonetheless, the repercussions of fire management in forests, are not well understood, with much to be explored.

\paragraph{}
%\paragraph{forest disturbance \\}

Forest dynamics are affected by various disturbances (e.g. fire, disease) \citep{attiwill_disturbance_nodate}. Disturbances play an important role in forest ecosystem, notably by generating heterogeneity in the landscape \citep{turner2010disturbance}. Disturbances can be defined by their duration \citep{perera_simulation_2015}. These range from disturbances that are considered instantaneous when compared to the timescale of forest growth (e.g. flood, windstorm, pest outbreaks), to long term ones, such as drought, temperature fluctuation or grazing.

%Also, some particular perturbations can be differentiate, the severe but rare events, this are termed "LIDS" for large and infrequent disturbances \citep{foster1998landscape}.
% rephrase if used this sentence. 

These different disturbances are often interrelated \citep{keane2015exploring}. This could create synergism between them \citep{mandre_environmental_2011} and have unanticipated responses \citep{perera_simulation_2015}. For example, fire and climatic fluctuations could interact to product cumulative effects \citep{romme_historical_2009}.
Another useful distinction is the implication of human in the disturbance, even if it is not possible to isolate anthropogenic perturbations from natural ones \citep{perera_simulation_2015}.
Indeed, the human impact on forests has increases dramatically over the last century. For example, the area logged per year in Canadian forests has doubled between 1960 and 1995 \citep{smith_canadas_2000}. This logging disturbance could be significantly different from the natural ones, when considering its impact on the ecosystem.

%\paragraph{Forest management \\}
\paragraph{}
For decades, sustainable forest management has been used to maintain forest ecosystems \citep{macdicken_global_2015}. This practice serves to maintain different aspect of the ecosystem such as productive functions, biological diversity, and socioeconomic functions \citep{makela_using_2012}. 
However, its main target is to conserve the forest ecosystem as an unified entity. %\citep{franklin1989toward}.
%Moreover, there is no unanimity on this different facet of sustainability \citep{martinez-vega_assessing_2016}.
At the same time, human demands from forests have broadened, making forest management a more complex endeavor \citep{eggers_balancing_2017}. Several recent studies have shown that ecosystem management should try to reproduce the nature disturbance regime in order to preserve the dynamics of the forest \citep{bengston_changing_1994, bengtsson_biodiversity_2000}. It could be possible to imitate the size, frequency and severity of disturbances \citep{hunter1988paleoecology, hunter1990wildlife}.

To reach these different targets (mainly sustainability and productive functions), various criteria are used. To be practical, such criteria needs to follow some guidelines: be easily measured, be sensitive to stress, be anticipatory (to counter change), be integrative (consider different facets of forest ecosystem such as soils and vegetation types) and have a low variability in its response \citep{dale_challenges_2001}. However, monitoring programs typically consider only few indicators and fail to take into account the complexity of the ecosystem \citep{dale_challenges_2001}.

%\paragraph{Fire \\}
\paragraph{}

One on the main disturbances in forests is fire, and in some regions, it is the most significant one. %Fires can create spatial patterns and heterogeneity in the landscape \citep{skinner_overview_nodate}. %Fires also affect plant behavior, such that plants develop traits for adaptation to fire (thick  bark  and fire-stimulated flowering, sprouting, seed release and/or germination) \citep{mckelvey1996overview, chang1996ecosystem}.
Fires are also linked with other perturbations, mostly climatic variation \citep{mckenzie_climatic_2004, da2018dynamics}, by its effects to on fuel \citep{schoennagel_interaction_2004} and by weather \citep{fernandes_fire-smart_2013}. Fuel is the dead wood, defined more precisely in the \hyperref[fuel]{methods}.

In practice, in some regions, fires are greatly affected by humans, and wildfires have been considerably reduced due to the intervention of firefighters \citep{fernandes_fire-smart_2013}. Different management practices are used, depending notably on the country and the tree species, and mostly based on the misconception that locking the dynamics of a system (decreasing its variability) will prevent it from collapse. On the other hand, in order to restore the natural dynamics of the forest, fires are sometimes allowed to run their course freely without intervention \citep{wallenius2011major}. 
Other forest fire preventions methods are the creation of a fuel break, also called \textit{defensible fuel profile zone} \citep{omi_effectiveness_nodate, adams2013mega}. The traditional rules of clear-cutting applies only partially, and \cite{bergeron_natural_2001} has argued that the situation is much more complex.

While in the past forest management was solely centered on the production of wood and timber, current forest management has been using a more integrative management approach, including economic, environmental, social and cultural dimensions \citep{eggers_balancing_2017, raison2001criteria}. Sustainable forest management has tried to maintain forest goods and services for the current and future uses \citep{macdicken_global_2015}. Management could also be based on natural disturbance in order to respect the inherent variability of the ecosystem \citep{bergeron_natural_2002}.

%\todo{explicit explanation about different strategies of fire management (e.g. inducing small-scale fires, thinning forests,...), and how successful or not they are.}

\newpage

\subsection*{Synopsis}
\addcontentsline{toc}{subsection}{Synopsis}


\paragraph{}
The main purpose of the present document is to demonstrate scenarios where  where different notions of ecosystem stability do not behave similarly. In order to do this, we look at how variability and collapse probability change with various model parameters, we consider different dynamical cases of the system, and describe the consequences for management practices.

\paragraph{}
In more detail, we begin by presenting a \hyperref[dynamical_system]{dynamical system} modelling the evolution of biomass under repeated fires. We adapt two well known measures \hyperref[variability]{\textit{variability}} and \hyperref[collapse_probability]{\textit{collapse probability}} to this model, and derive an analytic approximation to their value. We further demonstrate that the dynamical behavior of the forest-fire system can be understood along \hyperref[axes_definition]{two axes} of interest.

This allows us to categorise seven distinct dynamical states of the system, which we call \hyperref[dyna_cases]{cases}. We use out understanding of these different dynamical cases to decompose the different reactions to the \hyperref[impact_freq]{impact} of fire frequency on the two stability measures, and note that they can diverge. After studying the \hyperref[transition]{transition} from one dynamical case to another one with change of frequency, we show that the evolution of the two measures is not always synchronised. Focusing on a more pragmatic question, we show that \hyperref[fuel_removal]{\textit{fuel removal}} is major fire management strategy and that \hyperref[drought]{\textit{drought}} could have catastrophic consequences.  

We \hyperref[Discussion]{discuss} the major limitation of the present study, and examine the relevance of the distinctions of the two measures regarding the literature of early warning signals. We confirm the effectiveness of fuel removal and the danger of drought.
%\todo{adapt after the validation of the discussions}







%%%%%%%%%%%%%%%%%%%%%%%%%%%%%%%%%%%%%%%%%%%%%%%%%%%%%%%%%%%%%%%%%%%%%%%%%%%%%%%%%%%%%%%%%%%%%%%%%%%%%%%%
% Methods
%%%%%%%%%%%%%%%%%%%%%%%%%%%%%%%%%%%%%%%%%%%%%%%%%%%%%%%%%%%%%%%%%%%%%%%%%%%%%%%%%%%%%%%%%%%%%%%%%%%%%%%%

\newpage
\newpage
\section{Methods}


\subsection{Model}

\label{dynamical_system}

\paragraph{}
We consider a model of forest growth undergoing repeated fires, where $N$ represents the biomass of living trees, i.e. standing alive trees, and $W$ represents dead wood (for example dead wood debris and standing dead trees \citep{russell2015quantifying}. 

The living biomass $B$ follow a logistic growth \citep{tsoularis2002analysis, jensen1975comparison} with an Allee effect \citep{stephens1999allee, amarasekare1998allee}, in which biomass levels below a critical threshold will lead the system to collapse. The dynamics of the dead wood $W$ follow accumulation that is proportional to the density of $N$ and decay with time \citep{kahl_wood_2017, shorohova_stump_2012, christensen_estimation_1977, delaney_quantity_1998}. % \citep{barbosa_decomposition_2017} \citep{fravolini_quantifying_2018} \citep{wilson_dynamics_2005} \citep{zielonka_dynamics_nodate}. %% sinon ça fait trop de citation
Fire is modelled by discrete events and affects both $N$ and $W$, with a stronger effect for $W$ (dead wood burns more readily living trees \citep{brown1985predicting}). 
Finally, the severity (i.e. amount of living and dry biomass burned), is proportional to the density biomass $N$ and $W$ with stronger sensitivity to $W$ \citep{martinson_fuel_2013, safford_effects_2009, lecomte_effects_2006}. The model is given by equations \ref{eq:dim}.

\paragraph{}

\begin{subequations}\label{eq:dim}
 \begin{align}
\frac{dN}{dt} & =  gN(1-N/k)(N/a-1) - \delta_F(t)s(t)(N+\alpha W), \\
\frac{dW}{dt} & =  mN -dW - \beta\delta_F(t)s(t)(N+\alpha W).
 \end{align}
\end{subequations}

\paragraph{} % parameter 
%\todo{useful to give value ? (and justify with paper) it is not that simple, it could be a mess ...}
%\todo{all value at least positive non null, and m lower than max(dn / dt)}
In the first equation, $g$ is the growth rate of the living biomass, $k$ is the forest's carrying capacity (maximum living biomass). 
The Allee effect threshold $A$ represent the minimum quantity of living biomass in order to survive. In other words, if the value of $N$ goes below $a$, then the system collapses. The possible values of $a$ are between $0$ and $k$, but in practice $a$ takes low values, typically between $0.02 k$ and $0.2 k$. For the dynamics of dead wood, $m$ is the decay rate of living biomass into dead wood (the rate of conversion from $N$ to $W$, where $g$ is higher than $m$) and $d$ the decay rate of $W$ itself. 

For the fire term, $\beta$ is the coefficient that represent the fact that $W$ burns more easily than $N$. $\delta_F(t)$ and $s(t)$ represent, respectively, the fire occurrence and the severity of an eventual fire at time $t$. In other word, the occurrence of a fire is registered in the sequence $F$, where for a given time $t$, if $t\in F$ then a fire appears, the average fire frequency is denoted $f$. For a given fire, the strength follow an exponential law \citep{gauthier_les_2001, cyr_forest_2009} with an average of $s$. Finally, similarly to $\beta$, the parameter $\alpha$ represents the relative sensitivity of the fire strength to $W$, where both are assumed to be larger than $1$.

\paragraph{} % assumption
The major assumption of the model is that fire frequency is independent of $W$. As will be argued in the \hyperref[discussion_frequency_ass]{discussion}, this does not have a significant impact on the results presented. Another assumption is that a fire need dead wood ($W$) to burn. In other words, when $W=0$, fires cannot take place. This is based on the fact that fires are manly driven by fuel \citep{schoennagel_interaction_2004, stephens_effects_2012, syphard_comparing_2011, safford_effects_2009, stephens_experimental_2005}, i.e. $W$ in our model.


\paragraph{}
In order to simplify the study of the model, we adimensionnalise the system. As seen in equation \ref{eq:adim}, this leaves us with 7 parameters (see derivation in appendix \ref{adim}). 

\begin{subequations}\label{eq:adim}
 \begin{align}
\frac{dN}{dt} & =  N(1-N)(N-A) - \delta_F(t)s(t)(N+\alpha W), \\
\frac{dW}{dt} & =  mN -dW - \beta\delta_F(t)s(t)(N+\alpha W).
 \end{align}
\end{subequations}

\paragraph{} % give some 
We can remark that, with the absence of fire events, we have a \hyperref[equi]{bistable} system, i.e. there are two different stable states for the same parameter values. We further note that under these conditions, the fraction $\frac{m}{d}$ is the ratio of dead wood over alive wood. In practice, $N$ takes values in the range $[0,1]$ and $W$ in $[0, \frac{m}{d}]$.




%%%%%%%%%%%%%%%%%%%%%%%%%%%%%%%%%%%%%%%%%%%%%%%%%%%%%%%%%%%%%%%%%%%%%%%%%%%%%%%%%%%%%%%%%%
%%%%%%%%%%%%%%%%%%%%%%%%%%%%%%%%%%%%%%%% Measures %%%%%%%%%%%%%%%%%%%%%%%%%%%%%%%%%%%%%%%%
%%%%%%%%%%%%%%%%%%%%%%%%%%%%%%%%%%%%%%%%%%%%%%%%%%%%%%%%%%%%%%%%%%%%%%%%%%%%%%%%%%%%%%%%%%

%\newpage
\subsection{Measures}


\subsubsection{Definition}
\label{collapse_probability}
\paragraph{}
One of the main interests of forest management is avoiding system collapse. We therefore study the collapse probability of the forest ecosystem. In the present model, a collapse is reported when the density value of living biomass go below the Allee thresholds $a$. When it happens, the dynamics of both $N$ and $W$ will always converge to $0$. Measuring collapse probability requires running several simulation with different randomizations, where the collapse probability is the relative number of simulation in which the collapse has occurred. We mostly use simulations of overall time $T=100$, as this is a sufficiently long time for collapse events to take place, depending on various model parameters.
In Fig. \ref{fig:collapse}. we show a single simulation where a collapse has occurred within the simulation time.
%\todo{example when after the collapse, it remains $N$ and $W$, which keep decreasing, and strength the time series (ft = 300)} % here we can think it will always going to 0 because only of the fire.

\newpage

\begin{figure}[h!] 
\centering
\includegraphics[width=12.cm]{time_series_cp_1.png}
\caption{For $t$ close to $70$, a strong fire collapse the system. $N$ is put down below the Allee threshold effect (here $A=0.2$) and the dynamics converge now to the origin.
\label{fig:collapse}
}

\end{figure}

%However, it will depend on the time study used. Indeed, more the time study is long, the more is the risk to collapse. Even if it could be enough to used always the same time study, we present an improvement of this measure: probability to collapse by time unit. This upgrade is thus independent of the time study choose (details in \hyperref[proba_per_time_unit]{appendix}).


\label{variability}
\paragraph{}
A second stability measure of interest is variability. Variability is defined as the variance of a time series, where in this model we focus on the living biomass $N$, as it is of most interest. Considering variability is useful since forest management traditionally tends to lock the variability of the biomass \citep{bergeron_natural_2002}, i.e. management practices typically attempt to minimize the overall variability. Moreover, variability could be used to predict a critical change in the system \citep{karr_population_1982, pimm_risk_1988, bengtsson_predicting_1995}, which in the context of our model is a collapse. However other studies have shown no such results \citep{bengtsson_interspecific_1989, pollard_extinction_1992}. % or a negative relationship between ???\citep{lima_extinction_1996}. 


\paragraph{}
It is useful to consider the measure of variability as long as no collapse has occurred. This is both as once a collapse has occurred variability is no longer relevant, and since the collapse event and everything that follows will strongly skew variability values. This is demonstrated in Fig. \ref{fig:variability1}.

\begin{figure}[h!]
\centering
\includegraphics[width=12.cm]{time_series_sd_1.png}
\caption{Variability
\label{fig:variability1} 
}
\end{figure}


\paragraph{}
In practice, we have to make compromise to decrease different sources of possible bias. Even if the dynamics is initialised close to the average values taken by the variables (i.e. the approximations given by eq. \ref{NW1}), it is safer to not take into account the initial part of the time series, to compute collapse probability (thus we do not consider the first $10\%$ of the time series). Also, as explained previously, it is not useful to compute variability after a collapse occurs, but in order to use the same time to measure variability in all simulations, we restrict the computation to the second $10\%$ of the time series (see Fig. \ref{fig:variability2}). Indeed, because the likelihood to collapse in this short range time is low, the bias of collapse should be insignificant (algorithm in appendix \ref{algo_variability}). %Other ways of measuring variability have been implemented and are detailed in \hyperref[other_variability]{appendix}.

\begin{figure}[h!]
\centering
\includegraphics[width=12.cm]{time_series_sd_2.png}
\caption{Less biased computation of variability (the bias of collapse is less significant)
\label{fig:variability2} 
}
\end{figure}


\paragraph{}
A similar measure to variability is the coefficient of variability. Defined as the variability over the average value of the variabile. Indeed, a good measure of variability will be independent of the mean abundance if the dynamics are the same, but will not be independent if the dynamics change with mean abundance \citep{gaston_measurement_1993, noauthor_temporal_1994}. Here, because biomass value do not change dramatically, it is less significant to use the coefficient of variability.


\paragraph{}
Different techniques have been used to minimise the bias to variability estimation, due to collapse. However, the bias often persists \citep{seely2004complex}, which is why instead of running many simulation and average the effect, we propose to use only one simulation (only one set of fire events). One problem is that when we change the frequency, we need to choose between using the same time scale, so that the number of fire events will change (due to truncation) or using the same fire series but with a different time scale. Since both are relevant, both are computed, and compared, as shown in Fig. \ref{fig:same2} and Fig. \ref{fig:same1}.

%Variability analysis should be performed on data that are free from artefact \citep{seely2004complex} 
%\todo{More time means more variation \citep{lawton1988more} }

\begin{figure}[h!]
\centering
\includegraphics[width=12.cm]{same_2.png}
\caption{Same fire randomisation for different frequency and final time corresponding
\label{fig:same2}
}
\end{figure}

\begin{figure}[h!]
\centering
\includegraphics[width=12.cm]{same_1.png}
\caption{Same fire randomisation for different frequency and same final time (the first fire are the same, but the last are not necessarily present)
\label{fig:same1}
}
\end{figure}


\newpage
\subsubsection{Estimation}
\label{estimation}

\paragraph{} % estiamtion
Numerical approximation of measures have two main drawbacks. First, a significant amount of computer time is needed, especially if we want to have a robust approximation of the values, and if we want to compute measures for many different sets of parameters values. Moreover, numerical approximation of measures do not give direct information of which and how parameters affect the measures in question. We therefore derive an analytical estimation of both collapse probability and variability. 



\paragraph{}
\label{average_estimation}
To do this, we estimate first the average values of both $N$ and $W$. It could be remarked that it is also helpful for initialization of numerical simulations of the system, as it allows for shorter transitions times. 

An analytic expression is given in equation \ref{eq:NW1}, mainly based on the assumptions that fire frequency $f$ is high enough so that we can approximate the overall dynamics of the system as continuous (\hyperref[average]{see derication in appendix}).

\begin{equation} \label{eq:NW1}
\left\lbrace
\begin{array}{rcl}
N^{av} & = & \frac{1+a+\sqrt{(1-a)^2-4\gamma}}{2} \\
W^{av} & = & \epsilon N^{av} \\
\end{array}
\right.
\end{equation}

With,  
\begin{equation}   \label{eq:NW2}
\left\lbrace
\begin{array}{rcl}
\epsilon & = & \frac{m-\beta s f}{d + \beta s f \alpha} \\
\gamma & = & sf(1+\alpha\epsilon)
\end{array}
\right.
\end{equation}


\paragraph{}
As shown in Fig. \ref{fig:NWE}, this estimation works quite well, even for low values of frequency.

\begin{figure}[h!]
\centering
\includegraphics[width=9cm]{average.png}
\caption{Average values of $N$ and $W$, shown in green and brown, respectively. Solid lines shows values averaged from 10 simulations, while dashed lines shows values given by the estimation of eq. \ref{eq:NW1}. %Parameters used are: $a=0.2, ???$
\label{fig:NWE}
}
\end{figure}






\paragraph{} % estimation variability
The estimation of variability is based on the assumption that the two densities $N$ and $W$ remain close to their \hyperref[average_estimation]{estimated average}, $N^{av}$ and $W^{av}$. 

We denote by $\lambda$ the average severity of a fire, given by
\[
\lambda = min(\{s(N^*+\alpha W^*), \frac{W^*}{\beta s (N^*+\alpha W^*)}\}).
\]
Here the first values is simply the fire term, while the second values is the maximum severity of fire allowed by the fuel level, follows from the fact that when fuel is completely burned, the fire stop.

We make the assumption that the effects of fire are independent from each other (which is true when the growth dynamics are linear). Under this assumption, we can can consider the variability to be the linear response to a Poisson process  \citep{zelnik_impact_2018}, which leads us to  a prediction on the value of variability, as given by equation \ref{eq:varpred}. More details are given in appendix  \ref{variability_estimation}.
\begin{equation} \label{eq:varpred}
variability = f\frac{\lambda^2}{U}
\end{equation}


%\todo{figure for just this estimation of variability}
\begin{figure}[h!]
\centering
\includegraphics[width=10.cm]{variability_good.png}
\caption{Variability estimation
\label{fig:vargood}
}
\end{figure}

\paragraph{}
The precision of this estimation depends greatly on the parameters values. For example, we can see in Fig. \ref{fig:vargood} an example where the estimation worked well, while in Fig. \ref{fig:varbad} the estimation is less accurate. Specifically, while in Fig. \ref{dif:varbad} the general behaviour is captured well, the specific values are overestimated.
Generally, when the total severity of the fire is high, the accuracy of the estimation decreases. In practice, the product of the parameter $s.\alpha.\beta$ give an interesting clue of the precision of the variability estimation.
%Other estimation of variability have been implement and are presented in \hyperref[variability_estimation_other]{appendix}.

%\todo{make agian the plot with just the used estimation (just the "good" ones)}
\begin{figure}[h!]
\centering
\includegraphics[width=10.cm]{variability_bad.png}
\caption{Variability estimation
\label{fig:varbad}
}
\end{figure}


\paragraph{}
Similarly to variability, we also estimate the value of collapse probability. Here, we assume that system collapse as a result of a single fire event. This probability is called $cp1$. Derivation of $cp1$ could be found appendix \ref{cp_derivation}, where we consider the distribution of the strength and by using the estimation of $N^{av}$ and $W^{av}$ we deduce the risk to go below the Allee threshold effect
\[
\begin{array}{rcl}
cp1 & = & (N^*+\alpha W^*)((N^*-a+s)\exp(-\frac{N^*-a}{s}) - (\frac{W^*}{\beta}+s)\exp(-\frac{W^*}{s\beta})
\end{array}
\]
If we now use a binomial law, we can predict the overall collapse probability in a given time frame (where we use for simplicity the average number of fire events, as given by the average fire frequency $f$ and the simulation time $T$), given by equation \ref{eq:cp}
\begin{equation} \label{eq:cp}
\begin{array}{rcl}
cp & = & 1-(1-cp1)^{fT} \\
\end{array}
\end{equation}

\paragraph{}
Here again, the estimation work generally quite well, as shown in Fig. \ref{fig:CPE}, but the precision depends greatly on the specific set of parameters.
More generally, even if the maximum of the collapse probability is under estimated, the location of the peak is quite well predicted. As previously stated, the biases increase with the severity of the fire.

%\todo{do again the plot with a the left work well, and not very well for the right.}
\begin{figure}[h!]
\centering
\includegraphics[width=6.cm]{cp_good.png}
\includegraphics[width=6.cm]{cp_bad.png}
\caption{Cp estimation
\label{fig:CPE}
}
\end{figure}



\paragraph{}
In this study, we focus on two measures: variability and collapse probability. However as detailed in the introduction, several different notions of \hyperref[stability_litterature]{stability} are often considered in the literature. %These different aspects of stability are briefly presented in  \hyperref[stability_others]{appendix}.

\subsection{Classification of dynamical behaviour}

\label{axes_definition}

\paragraph{}
The aim of this section is to develop two separate axes, along which we ca asses the dynamical behaviour of the model, given the corresponding set of parameters.

\subsubsection{Fuel accumulation versus depletion}

\paragraph{}
The first axes focuses on representing the interplay between fuel accumulation (due to mortality of living wood) and fuel burning (by fires). In order to do this, we approximate the average of each effect at a specific value of ($N$,$W$). To do this we choose the point $(1, \frac{m}{2d})$, so as to capture the conditions when both processes should roughly equal each other. The choice $N=1$ is reasonable because $N$ is usually quite to $1$. We make the derivation when $W = \frac{m}{2d}$ because we can see if at this point the dynamics tends to pull back fuel to a lower level or up to a higher level. For example, if for a set of parameter, fuel accumulation dominates, then fuel is usually higher than $W = \frac{m}{2d}$ and if this density goes down at this point then dynamics will tends to increase the level of fuel.
%\todo{perhaps come back at the choice of "initial point" at he of demonstration to justify why it make sense}

\paragraph{}
We first want to estimate the fuel accumulation rate at the point $(1, \frac{m}{2d})$. We rewrite the second equation of the system (eq. \ref{eq:adim}) without the fire term (we denote $W_g$ as the solution of this new equation).
\[
\left\lbrace
\begin{array}{rcl}
\frac{d W_g}{dt} & = & mN-dW_g \\
W_g(0) & = & \frac{m}{2d} \\
\end{array}
\right.
\]
The solution is,
\[
\begin{array}{rcl}
W_g(t) & = & \frac{m}{d}(1-\frac{1}{2}\exp(-td)) \\
\end{array}
\]
And the fuel accumulation rate (at the point $(1, \frac{m}{2d})$) is: 
\[
\begin{array}{rcl}
\frac{dW_g}{dt}|_{t=0} & = & \frac{m}{2} \\
\end{array}
\]
The fuel accumulation rate is thus $\frac{m}{2}$.

\paragraph{}
We now want to approximate the fuel burning rate of $W$. The general fire term is:
\[
\delta_f(t)s(t)\beta(N+\alpha W)
\]
Applied at the point $(1, \frac{m}{2d})$ it is,
\[
\delta_f(t)s(t)\beta(1+\alpha\frac{m}{2d})
\]
We average the stochastic effect and have:
\[
fs\beta(1+\alpha\frac{m}{2d})
\]
The average fuel burning rate is thus $fs\beta(1+\alpha\frac{m}{2d})$.

\paragraph{}
We can now define the first axis as the ratio of the fuel \textit{Accumulation} rate divided by the average fuel \textit{Burning} rate.

\[
\begin{array}{rcl}
AB & = & \frac{fs\beta(1+\alpha\frac{m}{2d})}{\frac{m}{2}} \\
AB & = & fs\beta(\frac{2}{m}+\frac{\alpha}{d}) \\
\end{array}
\]
The parameter $AB$ thus tells us which process is more dominant, where low values of $AB$ (compared to $1$) means that accumulation govern the dynamics, and high value of $AB$ that depletion dominate.

\subsubsection{Possibility to collapse}


\paragraph{}
The second axis represent the possibility to collapse. A collapse is unlikely if the severity of the fire is substantially low (compared to $N = 1$). %If we take an arbitrary threshold of $0.01$ probability, then we have:
%We say that the system is unlikely collapse if the probability to collapse with only one fire is lower than $0.01$.
\[
\begin{array}{rccl}
P(s(N+\alpha W) > N-a ) & = & P(s(1+\alpha \frac{m}{d}) > 1-a ) \\
& = & P(s > \frac{1-a}{(1+\alpha \frac{m}{d})} ) \\
& = & \exp(-\frac{1-a}{s(1+\alpha\frac{m}{d})}) \\ 
\end{array}
\]


\paragraph{}
It is useful to note that for some  parameter values it is impossible to have a collapse from a single fire event because there is never enough fuel to maintain a fire strong enough to lead to the collapse of the forest. We will place this particular situation at the beginning of the axis.

%We now consider the equilibrium point $(1, \frac{m}{d}$) because it is when both $N$ and $W$ are higher and thus the total severity of the fire is higher too.

Here again we consider the system to be close to its equilibrium point $(1, \frac{m}{d}$).
\[
\left\lbrace
\begin{array}{rcl}
     N & = & 1 \\
     W & = & \frac{m}{d} \\
\end{array}
\right.
\]
Thus, the quantity burned in a single fire cannot be higher than $\frac{m}{d}$. \\
Formally,
\[
\begin{array}{crcl}
&s\beta(N+\alpha W) & < & W \\
\Rightarrow & s\beta(1+\alpha \frac{m}{d}) & < & \frac{m}{d} \\
\end{array}
\]
For the first equation (for $N$) we have a collapse only if
\[
\begin{array}{rccl}
                &  s(N+\alpha W) & > & N-a \\
\Leftrightarrow &  s(1+\alpha \frac{m}{d}) & > & 1-a \\ 
\Leftrightarrow &  \frac{m}{d\beta} & > & 1-a \\ 
\Leftrightarrow &  \frac{m}{d( 1-a)} & > & \beta \\ 
\end{array}
\]
Thus, if this condition is not respected (in practice when $\beta$ is high enough) it is impossible to have a collapse. We can see in Fig. \ref{fig:nevercollapse} an example where the lack of $W$ prevents the collapse of the forest.

\begin{figure}[h!]
\centering
\includegraphics[width=11cm]{return_never_1.png}
\caption{For this set of parameters, even for a strong fire, fuel is too low to maintain a fire from leading to a system collapse.}
\label{fig:nevercollapse}
\end{figure}


\paragraph{}
In conclusion, the second axis is constructed by the severity of the fire compared to others parameters, 
\[
\exp(-\frac{1-a}{s(1+\alpha\frac{m}{d})})
\]
and for some scenarios when the fuel is not sufficient to maintain a fire, when the following expression is not respected,
\[
\frac{m}{d( 1-a)} > \beta
\]
the scenario is constrained to be at the lower level of this axis.

\newpage
%%%%%%%%%%%%%%%%%%%%%%%%%%%%%%%%%%%%%%%%%%%%%%%%%%%%%%%%%%%%%%%%%%%%%%%%%%%%%%%%%%%%%%%%%%%%%%%%%%%%%%%%%%%%%%%%%%
% Result
%%%%%%%%%%%%%%%%%%%%%%%%%%%%%%%%%%%%%%%%%%%%%%%%%%%%%%%%%%%%%%%%%%%%%%%%%%%%%%%%%%%%%%%%%%%%%%%%%%%%%%%%%%%%%%%%%%

\section{Results}

\subsection{Model dynamics}



%a= 0.02 , m= 0.25 , d= 0.015625 , strength= 0.005 , alpha= 20 , beta= 2.0 % linear
%a= 0.02 , m= 0.25 , d= 0.015625 , strength= 0.01 , alpha= 40 , beta= 2.0 % counter
%a= 0.02 , m= 0.5 , d= 0.015625 , strength= 0.01 , alpha= 10 , beta= 2.0 % clock

\paragraph{}
We begin by considering the possible dynamics that can occur in our model. As seen in Fig. \ref{fig:dynexample}, by changing the value of the decay of dead wood $d$ (and in particular with the same fire frequency) we can see that the ecosystem behaviour can greatly change.

In the top panel, the recovery $N$ after a fire is slow and the level of fuel $W$ is typically high, which together often leads the system to a collapse. For the middle panel, while the recovery is still slow, the level of fuel is typically lower, so that most fires do not lead to a collapse. In the last illustration, the decay of $W$ is so fast that a single fire cannot lead to a collapse, and overall a collapse is quite unlikely, while the ecosystem typically recovers quite easily from any given fire. Hence, by changing only one parameter, we can have different dynamical behaviour, with corresponding consequences for variability and collapse probability.

\begin{figure}[h!]
\begin{center}
\includegraphics[height = 3.5cm]{results/time_series_2.png}
\includegraphics[height = 3.5cm]{results/time_series_3.png}
\includegraphics[height = 3.5cm]{results/time_series_4.png}
\end{center}
\caption{\label{fig:dynexample}Time series with different values of $d$: $0.0625$, $0.125$ and $0.25$}
\end{figure}
% or category or type, kinds 
% Time series for three different value of $m$ and $\alpha$
% the general behaviour is different ...
%Class linear, counter-clockwise and clockwise

\newpage

\subsection{Dynamics cases}
\label{dyna_cases}
%\paragraph{}
%In order to better understand the dynamics of the system, we distinguish different cases (and later subcases). Indeed, this facilitate the study, because we can so study case by case the different dynamics and their respective consequences.

%\paragraph{}
%A better differentiation could be done, it is possible to determine different cases of typical dynamical behaviour.
%%\paragraph{}
%Because we want to explore exhaustively the different dynamics cases, we introduce a coefficient to distinguish typical cases. Different \hyperref[other_ratio]{others coefficients} have been tested but only the following is used. After, another axes will be used to subdivide each cases.

\paragraph{}
Although we can see different behaviour by changing only one parameter, the connection remains unclear, and moreover, understanding the role of different parameters is quite difficult. We therefore choose to clarify this issue by using \hyperref[axes_definition]{previously defined} axes by which we can exhaustively compare all the possible types of dynamics.
    
The first axis compares the effect of fuel decay and burning. This axis is also used to construct the second axis: the possibility to collapse. From the combination of these two axes we can arrive at a number of cases of dynamical behaviour, as outlined in Fig. \ref{fig:universe}. We present below several time series for each such case, in order to elucidate the typical behaviour for each case.

\begin{figure}[h!]
\centering
\includegraphics[width=12cm]{results/time_series_each_cases.png}
\caption{Typical dynamical cases along the two axes \textit{accumulation VS depletion} and \textit{possibility to collapse}.}
\label{fig:universe}
\end{figure}

\newpage
\subsubsection{Accumulation VS depletion}

\paragraph{}
We choose two separate the first axis in three segments, by using two thresholds ($0.5$ and $20$) for the value of the ratio \textit{AB} (as defined in the methods section). The first part correspond to the cases when fuel accumulate substantially. In this case, fuel usually has time to come back to equilibrium ($W^{eq}=\frac{m}{d}$) before another fire occurs. Fig. \ref{fig:casere} shows typical behaviour for these cases.

\begin{figure}[h!]
\centering
\includegraphics[width=3.9cm]{return_to_eq_1.png}
\includegraphics[width=3.9cm]{return_to_eq_2.png}
\includegraphics[width=3.9cm]{return_to_eq_3.png}
\caption{Time series for different parameter sets corresponding to the cases of: Return to equilibrium}
%\label{fig:casere}
\end{figure}

\paragraph{}
At the opposite end, when the ratio \textit{AB} is high, fire tends to strongly limit the accumulation of $W$. This defines another group of cases, Depletion, with their typical behaviour shown in Fig. \ref{fig:casesdep}.
\begin{figure}[h!]
\centering
\includegraphics[width=3.9cm]{continue_1.png}
\includegraphics[width=3.9cm]{continue_2.png}
\includegraphics[width=3.9cm]{continue_3.png}
\caption{Time series for different parameter sets corresponding to the cases of: Depletion}
\label{fig:casesdep}
\end{figure}


\paragraph{}
Between the two previous groups, we define a third group called \textit{Fluctuation}. This happens when the effect of the fire and the growth are about equal in magnitude. In this situation, as seen in Fig. \ref{fig:casesfluc}, $W$ can take a larger range of value (from $0$ to $W^{eq}=\frac{m}{d}$).
\begin{figure}[h!]
\centering
\includegraphics[width=3.9cm]{middle_1.png}
\includegraphics[width=3.9cm]{middle_2.png}
\includegraphics[width=3.9cm]{middle_3.png}
\caption{Time series for different parameter sets corresponding to the cases of: Fluctuation}
\label{fig:casesfluc}
\end{figure}


\newpage
\paragraph{}
In order to visualise the limit of each group of cases, examples for the transitions between them are presented.
For the limit between \textit{Accumulation} and \textit{Fluctuation} (Fig. \ref{fig:casesmid1}) we can see that a fire often occurs before the dynamics come back to equilibrium. For the second one (Fig. \ref{fig:casesmid2}), we see that fuel is quite low, but could sometimes take relatively high value.

\begin{figure}[h!]
\centering
\includegraphics[width=3.9cm]{lim_eq_middle_1.png}
\includegraphics[width=3.9cm]{lim_eq_middle_2.png}
\includegraphics[width=3.9cm]{lim_eq_middle_3.png}
\caption{Time series for different parameter sets corresponding to the limit between Accumulation and Fluctuation.}
\label{fig:casesmid1}
\end{figure}


\begin{figure}[h!]
\centering
\includegraphics[width=3.9cm]{lim_c_middle_1.png}
\includegraphics[width=3.9cm]{lim_c_middle_2.png}
\includegraphics[width=3.9cm]{lim_c_middle_3.png}
\caption{Time series for different parameter sets corresponding to the limit between Depletion and Fluctuation.}
\label{fig:casesmid2}
\end{figure}

%%%%%%%%%%%%%%%%%%%%%%%%%%%%%%%%%%%%% subcases %%%%%%%%%%%%%%%%%%%%%%%%%%%%%%%%%%%%%%%%%%%%%%%%%%

\newpage

%\subsubsection{"Accumulation"}
\subsubsection{Possibility to collapse}


\paragraph{} %\todo{expand ? it is the main paragraph of the subsubsection ... }
After using the axis \textit{AB} to distinguish between groups of cases, we now use the axis \textit{possibility to collapse} (see methods) to further separate them into specific dynamical cases. In general, a separation into three parts could be done, by using two thresholds ($0.01$ and $0.2$) to the value of the axis. The first part, called \textit{Unlikely collapse}, comprises the lower value of the axis where the condition $\frac{m}{d( 1-a)} > \beta$ is not respected. The transitory part called "Possible collapse" when the collapse is unpredictable. And finally, the \textit{Inevitable collapse} when any given fire is likely to lead to the system's collapse.

Using the two axes together allows us to have a grid of typical dynamical cases, as shown in Fig. \ref{universe}. However, the number of these cases is $7$ and not $9$, when for the case-group $Depletion$ a collapse in always unlikely, leading to only one case in this group. In other words, if the ratio \textit{AB} is high, the level of fuel is too low to allow a fire to collapse the system. We therefore do not expect to have the case \textit{Depletion - Possible collapse}, and certainly not the case \textit{Depletion - Inevitable collapse}. Nonetheless, let us recall that even is the case \textit{Unlikely collapse}, the system is still able to collapse, but rather the risk is very low.


\paragraph{Accumulation - Unlikely collapse\\}
Fig. \ref{fig:caseauc} shows two times series, with two different sets of parameters corresponding to the case \textit{Accumulation - Unlikely collapse}. The left instance corresponds to a situation where the system does not collapse because the condition $\frac{m}{d( 1-a)} > \beta$ is not respected. As can be seen, even a fire that is large enough to deplete $W$ entirely cannot lead to the collapse of $N$. The right instance corresponds to low severity of the fire, where usually both $N$ and $W$ have enough time to come back to equilibrium between two fires.

\begin{figure}[h!]
\centering
\includegraphics[width=6cm]{return_never_1.png}
\includegraphics[width=6cm]{return_never_2.png}
\caption{Time series for different parameter sets corresponding to the case: Accumulation - Unlikely collapse.}
\label{fig:caseauc}
\end{figure}

\paragraph{Accumulation - Inevitable collapse\\} % always
For some sets of parameter, the system is certain to collapse. Of course, this occurs when the condition $\frac{m}{d( 1-a)} > \beta$ is respected. Moreover, the severity of the fire needs to be strong enough to lead to a collapse. Because the group \textit{Accumulation} implies that the dynamics usually return to equilibrium between two fires, the system could collapse only by a single strong fire, as we can see below.

\begin{figure}[h!]
\centering
\includegraphics[width=3.9cm]{return_always_1.png}
\includegraphics[width=3.9cm]{return_always_2.png}
\includegraphics[width=3.9cm]{return_always_3.png}
\caption{Time series for different parameter sets corresponding to the case: Accumulation - Inevitable collapse.}
%\label{fig:universe}
\end{figure}


%\todo{préciser que le temps d'étude n'est pas tjrs suffisant pour le constater}
\paragraph{Accumulation - Possible collapse \\} % between
Between these two extremes cases, there exists a range of parameters when a collapse may occur (Fig. \ref{caseapc}). This happens when $W$ is high enough and also when the actual strength of the fire sufficiently large. We could remark that this case depends greatly on the final simulation time used, because, while this system is able to collapse, its needs a strong enough fire which is not common. In other words, for a longer time study, the system will have more risk of collapse, because the probability to have a strong fire would be higher.

\begin{figure}[h!]
\centering
\includegraphics[width=3.9cm]{return_between_1.png}
\includegraphics[width=3.9cm]{return_between_2.png}
\includegraphics[width=3.9cm]{return_between_3.png}
\caption{Time series for different parameter sets corresponding to the case: Accumulation - Possible collapse.}
\label{fig:caseapc}
\end{figure}

%%%%%%%%%%%%%%%%%%  \todo{... link with cp (but after) ... } 

\paragraph{Depletion \\}
When $W$ is always low, a collapse is highly unlikely. In practice, we can observe a collapse only when the set of parameter is close to the cases of \textit{Fluctuation}. Is is then that $W$ can get to higher enough values to generate a substantial fire, but this is rarely observed. In Fig. \ref{cased} we can observe that $W$ comes back to $0$ with each fire and does not have time to accumulate, and thus lead to a strong fire.
%\todo{jamais collapse, expliquer pourquoi avec un très simple calcul}
%\todo{nuancer dans le cas ou on est proche du cas "equivalence" and fuel is not always low enough}

\begin{figure}[h!]
\centering
\includegraphics[width=6cm]{continue_1.png}
\includegraphics[width=6cm]{continue_2.png}
\caption{Time series for different parameter sets corresponding to the case: \textit{Depletion}.}
\label{fig:cased}
\end{figure}


%\subsubsection{"fluctuating"}
%\todo{extrapoler, on divise en 2 cas pour s'approcher des 2 autres cas}
%\todo{quand proche continuous, collapse happen if fuel go too high}
%\todo{illustrate}

\newpage
\paragraph{fluctuation \\}
The cases in the group \textit{Fluctuation} are harder to study, mainly because $W$ values are are strongly vary, and it is thus more difficult to have analytic results. Depending on the value of the ratio \textit{AB}, it could make sense to assimilate this case to the case \textit{Accumulate} or \textit{Depletion}.

\paragraph{}
When $AB$ is close enough to the limit with the case \textit{Depletion}, we could consider that the level of $W$ remains low enough most of the time, in order to avoid collapse. However, the risk to collapse here is not negligible, as can be seen in Fig. \ref{fig:casefbyd}.

\begin{figure}[h!]
\centering
\includegraphics[width=5.5cm]{equivalent_high_1.png}
\includegraphics[width=5.5cm]{equivalent_high_3.png} \\
\includegraphics[width=5.5cm]{equivalent_high_2.png}
\caption{Time series for different parameter sets corresponding to the case: \textit{Fluctuation}, but close to \textit{Depletion}.}
\label{fig:casefbyd}
\end{figure}

\newpage
\paragraph{}
On the other hand, when $AB$ is lower, we again have three similar cases to the \textit{Accumulation} group. We can observe in Fig. \ref{fig:case3fluc} that in this situation, the dynamics do not greatly change.%, and that the extrapolation could make sense (but only if $AB$ is low enough).
%\todo{ratio low, proche "return", on herite des 3 cas, never always, between}
%\todo{illustrer 3 exemples pour chaque cas}

%\paragraph{}

\begin{figure}[h!]
\centering
\includegraphics[width=6cm]{equivalent_low_never_2.png}
\includegraphics[width=6cm]{equivalent_low_never_3.png} \\
\includegraphics[width=6cm]{equivalent_low_between_2.png}
\caption{From left to right: Fluctuation - Unlikely collapse, Fluctuation - Possible collapse and Fluctuation - Inevitable collapse.}
\label{fig:case3fluc}
\end{figure}

%\begin{figure}[h!]
%\centering
%\includegraphics[width=3.9cm]{equivalent_low_never_1.png}
%\includegraphics[width=3.9cm]{equivalent_low_never_2.png}
%\includegraphics[width=3.9cm]{equivalent_low_never_3.png}
%\caption{Time series for different parameter sets corresponding to the case: fluctuating - unlikely collapse}
%\label{fig:universe}
%\end{figure}


%\paragraph{}

%\begin{figure}[h!]
%\centering
%\includegraphics[width=3.9cm]{equivalent_low_always_1.png}
%\includegraphics[width=3.9cm]{equivalent_low_always_2.png}
%\includegraphics[width=3.9cm]{equivalent_low_always_3.png}
%\caption{Time series for different parameter sets corresponding to the case: fluctuating - possible collapse}
%\label{fig:universe}
%\end{figure}

%\paragraph{}

%\begin{figure}[h!]
%\centering
%\includegraphics[width=6cm]{equivalent_low_between_1.png}
%\includegraphics[width=6cm]{equivalent_low_between_2.png}
%\caption{Time series for different parameter sets corresponding to the case: fluctuating - inevitable}
%\label{fig:universe}
%\end{figure}


\newpage
\subsection{Connection between variability and collapse probability}

\label{impact_freq}

\paragraph{}
One major focus of this study is to look at the impact of frequency over variability and collapse probability. We study this for each of the seven dynamical cases defined previously, with an appropriate range of frequency values.

\paragraph{}
We begin by looking at the dynamical case called \textit{Accumulation - Unlikely collapse}, as shown in Fig. \ref{fig:vcp1}. The red background is used to mark the dynamical case \textit{Accumulation}, while the yellow corresponds to \textit{Fluctuation}. We can see that the variability increases with the frequency, however, collapse probability remains negligible. We can remark that the estimation (the continuous line) work reasonably well in this situation.

\begin{figure}[h!]
\begin{center}
\includegraphics[width=9cm]{results/measures_acc_unlikely.png}
\end{center}
\caption{\label{fig:vcp1}Variability and collapse probability as a function of fire frequency for the dynamical cases: Accumulation - Unlikely collapse and Fluctuation - Unlikely collapse.}
\end{figure}


\paragraph{}
If we start with a fluctuation case for low frequencies, as in Fig. \ref{fig:vcp2}, the collapse probability is still zero. Here the variability first increases, following the behaviour of the case \textit{Accumulation - Unlikely collapse} and decrease for higher frequencies, when we approach the \textit{Depletion} dynamical case.

\begin{figure}[h!]
\begin{center}
\includegraphics[width=9cm]{results/measures_fluctu_unlikely.png} \\
\end{center}
\caption{\label{fig:vcp2}Variability and collapse probability as a function of fire frequency for the dynamical cases: Fluctuation - Unlikely collapse.}
\end{figure}

\paragraph{}
When frequency is high enough, $AB$ take high values, and the fuel is constrained to low values. This means that the two measures are quickly decreasing. Depending on the parameter set, collapse probability could already be zero, or crash in the Depletion dynamical case (Fig. \ref{fig:vcp3}).

\begin{figure}[h!]
\begin{center}
\includegraphics[width=9cm]{results/measures_depletion.png}
\end{center}
\caption{\label{fig:vcp3}Variability and collapse probability as a function of fire frequency for the dynamical cases: Depletion.}
\end{figure}


\paragraph{}
For the two dynamical cases \textit{accumulation - possible collapse} and \textit{accumulation - inevitable collapse} both variability and collapse probability are increasing (Fig. \ref{vcp4}). To clarify, even in the dynamical case \textit{Accumulation - Inevitable collapse} the collapse probability could be low because there isn't enough fire in the time study to have a substantial risk of system collapse.

\begin{figure}[h!]
\begin{center}
\includegraphics[width=6cm]{results/measures_acc_possible.png}
\includegraphics[width=6cm]{results/measures_acc_inevitable.png} 
\end{center}
\caption{\label{fig:vcp4}Variability and collapse probability as a function of fire frequency for different dynamical cases. Left \textit{Accumulation - Possible collapse}; Right \textit{Accumulation - Inevitable collapse}.}
\label{acc_possible_and_inevitable}
\end{figure}

\paragraph{}
For the two dynamical cases \textit{Fluctuation - Possible collapse} and \textit{Fluctuation - Inevitable collapse}, both variability and collapse probability first increase and later decrease at higher frequencies (Fig. \ref{vcp5}). 

\begin{figure}[h!]
\begin{center}
\includegraphics[width=6cm]{results/measures_fluctuation_possible.png} \includegraphics[width=6cm]{results/measures_fluctuation_inevitable.png}
\end{center}
\caption{\label{fig:vcp5}Variability and collapse probability as a function of fire frequency for different dynamical cases. Left \textit{Fluctuation - Possible collapse}, Right \textit{Fluctuation - Inevitable collapse}}
\end{figure}



\newpage
\label{transition}
\paragraph{}
Within each dynamical case, frequency affect both variability and collapse probability. Because $AB$ depends on frequency, changing this parameter could imply a transition between dynamical cases. In order to study the effect of frequency across dynamical cases, we study the transition from one dynamical case to another. To do this, we compute the probability to change the dynamical case if the frequency is doubled. This gives us the transition matrix shown in table \ref{tab:transition}.

%\paragraph{}
%Previously, the frequency range used was choose to remains in the same sub cases. Now, we want to know what happens when the frequency range is higher. To do this, we study the transition of subcases when we double the frequency. 
%We can see that the ability to collapse in one fire do not change, however, the ratio change, and thus, doubling the frequency could change the dynamical cases for axis 1 but not for axis 2. \todo{reformulate}


\begin{table}[h!]
    \centering
    \begin{tabular}{|c|c||c|c|c|c|c|c|c|}
        \hline
        $AB$ && \multicolumn{3}{c|}{accumulation} & \multicolumn{3}{c|}{fluctuation} & \\
        \hline
        & \rotatebox{45}{possibility to collapse} & \rotatebox{90}{unlikely collapse} & \rotatebox{90}{possible collapse} & \rotatebox{90}{inevitable collapse} & \rotatebox{90}{unlikely collapse} & \rotatebox{90}{possible collapse} & \rotatebox{90}{unlikely collapse} & \rotatebox{90}{depletion} \\
        \hline
        \hline
        \multirow{3}*{accumulation} & unlikely collapse & 0.757 & 0 & 0 & 0.243 & 0 & 0 & 0 \\
        \cline{2-9}
        & possible collapse & 0 & 0.167 & 0 & 0 & 0.833 & 0 & 0 \\
        \cline{2-9}
        & inevitable collapse & 0 & 0 & 0.743 & 0 & 0 & 0.257 & 0 \\
        \hline
        \multirow{3}*{fluctuation} & unlikely collapse & 0 & 0 & 0 & 0.838 & 0 & 0 & 0.162 \\
        \cline{2-9}
        & possible collapse & 0 & 0 & 0 & 0 & 0.857 & 0 & 0.143 \\
        \cline{2-9}
        & unlikely collapse & 0 & 0 & 0 & 0 & 0 & 0.827 & 0.173 \\
        \hline
        depletion && 0 & 0 & 0 & 0 & 0 & 0 & 1 \\
        \hline
    \end{tabular}
    \caption{Transition matrix}
    \label{tab:transition}}
\end{table}

\paragraph{}
We focus on the general effect of changing frequency on the two measures, variability ad collapse probability, across the different dynamical cases. This is presented in Fig. \ref{fig:loops}, which shows four different possible outcomes to increasing fire frequency on the change in both measures.
In general we see that for the extremes (very low and very high frequencies) both measures are very small, and they are higher for an intermediate value of frequency.

%Depending on the parameter set choose, the dynamics is different and therefore, the variability and the collapse probability do not react similarly. On the first one, variability and cp first increase together and after decrease (the decreasing part is not always present). On the second one, the two measure are also increasing and decreasing but not in the same time, more surprisingly cp increase before variability. However, in the last class presented, the loop is clockwise.

\paragraph{}
Depending on the parameter set chosen, the overall trajectory can take markedly different shapes in the variability-collapse probability plane. In the top-left panel, collapse probability and variability initially increase together, but collapse probability start decreasing for lower frequencies, creating a circular loop shape. For the top-right panel, as well as the bottom-left one, the loop structure us quite different, in particular since collapse probabilities to get high values. In particular, for the bottom-left case the relation between the two measures is almost linear. Finally, in the bottom-right panel we see a clockwise loop (i.e. variability decreases before collapse probability does so), which is relatively uncommon.

\begin{figure}[h!]
\begin{center}
\includegraphics[width=6cm]{case_circle.png}
\includegraphics[width=6cm]{case_triangular.png}
\includegraphics[width=6cm]{case_linear.png}
\includegraphics[width=6cm]{case_clockwise.png}
%\includegraphics[width=10cm, height = 4.2cm]{case_linear.png}
%\includegraphics[width=10cm, height = 4.2cm]{case_triangular.png}
%\includegraphics[width=10cm, height = 4.2cm]{case_clockwise.png}
\end{center}
\caption{Different examples of the relationship of variability and collapse probability. % or category or type, kinds
\label{fig:loops}}
\end{figure}


%\todo{plot in results the graph with var and cp increasing with same fire and same time, to validate the results}
%\todo{precise that counter clock est assez rare}


\newpage
\subsection{Implication for perturbations effects.}


\subsubsection{fuel removal}
\label{fuel_removal}
\paragraph{}
We now want to study the impact of the parameter $d$, the decay of dead wood, on the variability and collapse probability of the system. To do this, we repeat the study for three groups of dynamical cases (we use only the main axis of \textit{Accumulation VS Depletion}). This is since the second axis is itself related to the notion of collapse probability, and therefore it is more straightforward to directly measures the collapse probability from simulation than to estimate it. In practice, we choose for each case a set of parameters so that we can vary the value of $d$ while remaining in the same dynamical case.

We can see that for each dynamics case, increasing $d$ decreases variability and collapse probability (Fig. \ref{fig:changed}). This is intuitive, because increasing $d$ decreases the value of $W^{av}$, the average of dead wood in the system.  If the decay of the dead wood is higher, the density of the dead wood tends to be lower. The consequences is that the fire no longer has enough fuel to trigger a collapse and the variability is lower because $W$ always takes small values.

However, if in the case \textit{Accumulation} variability and collapse probability are similar, while for the case \textit{Fluctuation} variability decreases slower than collapse probability. Also, in the case \textit{Depletion}, the variability decreases even if the collapse probability remains zero.

\begin{figure}[h]
\begin{center}
\includegraphics[width=6cm]{results/return_to_equilibrium_1.png}
\includegraphics[width=6cm]{results/equivalent_1.png}
\includegraphics[width=6cm]{results/fuel_low_1.png}
\end{center}
\caption{\label{fig:temp}Impact of $d$, from left to right: \textit{Accumulation}, \textit{Fluctuating} and \textit{Depletion}. \label{fig:changed}}
\end{figure}


\newpage
\subsubsection{Drought}
\label{drought}



\paragraph{}
We now focus on the drought scenarios. This is a current subject in ecological research, with many recent papers studying this topic. The main focus is on the consequences of extreme weather as a result of climate change. 

It has been shown that drought increases the mortality rate of the forest \citep{bradstock_biogeographic_2010}. In our model, this corresponds to an increase in the parameter $m$. However, the main consequences of drought is an increase of the strength of fire, notably because the alive wood is drier. To simulate this, we increase  the parameters $s$, $\alpha$ and $\beta$, where we increase $s$ twice faster than the other two parameters. According to \cite{fernandes_fire-smart_2013, fairman_too_2016} fire frequency increase with drought, however, \cite{bergeron_predicting_nodate} claimed the opposite. In the present study, we choose to not change the fire frequency.

%\paragraph{}
%It can be \hyperref[drought_increase]{demonstrated} that increasing this $4$ parameters increase the ratio $AB$. Thus, a drought scenarios will tend firstly to increase both variability and the collapse probability and after a decrease of both measures.

\paragraph{}
In the Fig. \ref{fig:drought1}, \ref{fig:drought2}, \ref{fig:drought3} \ref{fig:drought4}, the consequences of drought for each dynamical case are presented. It can be seen that variability can increase without collapse probability changing (for the dynamical case \textit{Accumulation - Unlikely collapse} and \textit{Fluctuation - Unlikely collapse}). For other cases (\textit{Accumulation - Possible collapse} and \textit{Accumulation - Inevitable collapse}) variability does not follow a clear trend, and the collapse probability tends to have a same behaviour or increasing further. The two dynamical cases \textit{Fluctuation - Possible collapse} and \textit{Fluctuation - Inevitable collapse} show an increase of first collapse probability and later variability. In general, drought increases the collapse probability, but an increase in the variability does not always occur.


\begin{figure}[h]
\begin{center}
\includegraphics[width=6cm]{results/drought/return_never.png}
\includegraphics[width=6cm]{results/drought/equivalent_never.png}
\end{center}
\caption{\label{fig:temp}Effect of drought for the dynamical cases: \textit{Accumulation - Unlikely collapse} at the left and \textit{Fluctuation - Unlikely collapse} at the right.
\label{fig:drought1}} 
\end{figure}

\begin{figure}[h]
\begin{center}
\includegraphics[width=6cm]{results/drought/return_moderate.png}
\includegraphics[width=6cm]{results/drought/return_always.png}
\end{center}
\caption{\label{fig:temp}Effect of drought for the dynamical cases: \textit{Accumulation - Possible collapse} at the left and \textit{Accumulation - Inevitable collapse} at the right.
\label{fig:drought2}}
\end{figure}

\begin{figure}[h]
\begin{center}
\includegraphics[width=6cm]{results/drought/equivalent_moderate.png}
\includegraphics[width=6cm]{results/drought/equivalent_always.png}
\end{center}
\caption{\label{fig:temp}Effect of drought for the dynamical cases: \textit{Fluctuation - Possible collapse} at the left and \textit{Fluctuation - Inevitable collapse} at the right.
\label{fig:drought3}}
\end{figure}

\begin{figure}[h]
\begin{center}
\includegraphics[width=9cm]{results/drought/fuel_low.png}
\end{center}
\caption{\label{fig:temp}Effect of drought for the dynamical case: \textit{Depletion}. \label{fig:drought4}}
\end{figure}





%%%%%%%%%%%%%%%%%%%%%%%%%%%%%%%%%%%%%%%%%%%%%%%%%%%%%%%%%%%%%%%%%%%%%%%%%%%%%%%%%%%%%%%%%%%%%%%%%%%%%%%%%%%%%%%%%%
% Discussion
%%%%%%%%%%%%%%%%%%%%%%%%%%%%%%%%%%%%%%%%%%%%%%%%%%%%%%%%%%%%%%%%%%%%%%%%%%%%%%%%%%%%%%%%%%%%%%%%%%%%%%%%%%%%%%%%%%


\newpage
\section{Discussion}
\label{Discussion}

\paragraph{} % brief summary,  recall objective ...
In this study we have focused on a model of forest dynamics undergoing repeated fires. We have seen that it is possible to distinguish between several types of dynamics in this model, each having a characteristic behaviour. For each such case, the two measures considered, variability and collapse probability, do not react similarly when frequency increases. Overall these two measures are sometimes dissimilar, as they do not react correspondingly to different external forcing. This could be a problem if we want to use variability as a tool to stabilise the dynamics, or as indicators of critical transitions.

\subsection{Limitations}

\paragraph{}
The present study aims to reveal different properties of a forest-fire system, and more generally, of ecosystems where the perturbation regime is strongly coupled to the state of the system itself. For instance, we found that \textit{variability} can not always be used to predict a collapse. However, more work is needed if this approach is to become predictive.

\label{discussion_frequency_ass}
We have assumed that the frequency of the fire is independent of the density of biomass. It can be argued that fuels have an influence on both severity and frequency  of fires \citep{schoennagel_interaction_2004}. However, adding this feedback ($W$ influencing the frequency) will surely tends to decrease the density biomass of $W$ (because, when $W$ is higher, the frequency is higher too, and so the dynamics keep a low value of $W$). In other words, adding this feedback will lead to the \textit{depletion} scenarios being more common.

Another issue is that we only consider one kind of dead wood, while it is often useful to consider several different types of such material (e.g. coarse woody debris, fine woody debris, below ground) \citep{russell_quantifying_2015}. In the literature, data is rarely available for all these types, and some wood types burn more easily than others. Moreover, in practice we can distinguish between several fire regimes (e.g.  crown fires, severe surface fires, and light surface fires) \citep{reichle_fire_1981}, where different fires regimes have different sensitivities to types of dry wood. Therefore the dynamics of fires may be better modelled in a more complex manner, which should change the way the feedback mechanisms works in this ecosystem.

A more minor issues is that we only consider a non-spatial biomass density. This can be an issue since the spatial distribution of both trees and past fires can affect fire propagation (especially for small fires) \citep{beaty_spatial_2002}. For example, a burned area can create an obstacle when another fire occurs \citep{bergeron_natural_2002, ager_modeling_2007}. It is sometimes used as a management tool called a \textit{fuel break} to prevent large fires, where a manager creates a continuous region with no fuel, that will stop the propagation of fires \citep{syphard_comparing_2011, agee_use_2000}.

%Data varies greatly and also depend on several variable (type of forest, localisation ... )

%\todo{talk again about the bias from cp to variability ??} 

\subsection{Mathematical/modelling aspects}

\paragraph{}
Because the collapse of the system affects the variability within the system, it is often difficult to have robust computation of the variability. Constructing an unbiased computation of variability does not seem to be possible. Specifically, because a collapse can occur at all times, it is difficult to calculate the variability under no collapse without introducing a bias to the calculation.

It could be argued that the dissimilarity between variability and collapse probability is an artefact of this bias. In particular, the peak of variability could arise from the collapse events. To validate that this is not the case, we use two different methods of comparing simulations with different parameters (presented in \hyperref[same_fire]{methods}). Specifically, we use either the same time scale or the same set of fire events to compute variability (before the time of the first collapse), allowing us to show that the peak in variability is not directly related to collapse events. 

As seen in Fig. \ref{fig:samefire}, using both methods we have a peak in variability, allowing us to conclude that the peak of variability is not an artefact of the collapse. In particular, for the same-fire method, we have a smooth peak forming, hinting that the source of the non-monotonous behaviour is not in discrete events (and thus not related to a collapse). Additionally, we note that it is possible to have the peak in variability before a peak in collapse probability, hinting further that the two need not be connected.

\begin{figure}[h]
\begin{center}
\includegraphics[width=12cm]{same_fire.png}
\end{center}
\caption{\label{fig:samefire}Variability evolution with frequency, computed in a robust way, using either exactly the same fires sequence (\textit{same fire}) or the same first fire where all simulations end at the same time for all frequencies (\textit{same time}).}
\end{figure}


\subsection{Collapse indicators}

%\paragraph{}
%In ecology, as well as in other applied sciences, an essential care is given to the 

\paragraph{}
We have seen in the \hyperref[def_detection]{introduction} that the literature about detection of critical transitions is mainly based on the closeness to a tipping point. The model we consider is a bistable system, so that naively these indicators could be thought to easily apply. However, the collapse scenarios we consider typically do not occur close to the critical point, and hence are not directly applicable using the notion of critical slowing down. On the other hand, since the system cannot recover once a collapse has occurred, flickering does not occur either, and hence methods related to this notion are not clearly applicable either.

However, we do find that variability can sometimes be a good indicator of system collapse. In particular, for two dynamical cases, \textit{accumulation - possible collapse} and \textit{accumulation - inevitable collapse}(Fig. \ref{acc_possible_and_inevitable} at the left and at the right, respectively), this may apply. Conversely, for the dynamical case \textit{accumulate - unlikely collapse}, variability increase without an increasing of the collapse probability. More generally, Fig. \ref{fig_loop} shows that for different sets of parameters it is possible to have different relations between the two measures. One measure could increase (respectively decrease) before the other increases (respectively decreases). %When the loops shown in Fig. \ref{fig_loop} turn in two different directions (mostly counter clockwise) the collapse probability increases and reaches a peak before the variability.


\subsection{Management}

\paragraph{}
%Traditional fire management privileges fire suppression \citep{fernandes_fire-smart_2013}. 
Modern firefighting strategies have often developed unnatural fuel accumulation, and therefore more severe wildfires \citep{schoennagel_interaction_2004}. The present study confirms that fuel accumulation is indeed the main driver of fire damage. Decreasing the frequency of fire is a misconception called the firefighting trap \citep{collins_forest_2013}, and more often other methods should be applied.

Fuel treatment is typically seen as the best strategy to prevent large wildfire \citep{liu_studying_2013, martinson_performance_nodate}. One specific method is to use prescribed burning \cite{liu_analyzing_2010}. In our methodology, we could model this by increasing the fire frequency, which at some point will drive us to transition into the \textit{depletion} dynamical cases, where fuel level are too low to lead to a collapse. It have been argued by \cite{scholl_fire_2010} that managers have to use multiple burns at short intervals. While it could be argues that multiple prescribed burning would adversely affect the ecosystem, it have been demonstrated that this impact is lower than for wildfires \citep{alcaniz2018effects, fultz2016forest, wiedinmyer2010prescribed}. 

Prescribed burning could be used jointly with thinning, which would allow to both treat canopy and surface fuels \citep{kalies_tamm_2016, agee_basic_2005}. Thinning treatments are mostly aimed at removing a few trees to increase the growth, health and value of the other trees. This strategy is often efficient \citep{hurteau2008carbon} and could also better resist droughts \citep{d2013effects} and pest outbreaks \citep{waring2005silvicultural}. In the present study, we could see that decreasing the value of the parameter $d$ significantly decreases the risk to collapse.

%\subsection{Last words}
%\todo{perhaps}







\newpage
%%%%%%%%%%%%%%%%%%%%%%%%%%%%%%%%%%%%%%%%%%%%%%%%%%%%%%%%%%%%%%%%%%%%%%%%%%%%%%%%%%%%%%%%%%%%%%%%%%%%%%%%%%%%%%%%%%
% Conclusion
%%%%%%%%%%%%%%%%%%%%%%%%%%%%%%%%%%%%%%%%%%%%%%%%%%%%%%%%%%%%%%%%%%%%%%%%%%%%%%%%%%%%%%%%%%%%%%%%%%%%%%%%%%%%%%%%%%

\section*{Conclusion}
\addcontentsline{toc}{section}{Conclusion}

\paragraph{}
The aim of the present document is to demonstrate the disparity between two well used measures of dynamical systems: variability and collapse probability. We used a particular model of a forest fire, to demonstrate the interplay between these two measures . By constructing a partition of different dynamical behaviour, we could show that these measures behave differently when changing a parameter. Although the two measures are not used in a predictive way, it is established that there is a clear dissimilarity between them.

In the specific context of forest-fire dynamics, we show that the notion that forest managers should act on fire frequency (by deploying firefighters or igniting counter fires) in order to diminish it is a misconception that can put the system in danger (by increasing the collapse probability). This is because fuel accumulates and allows a large fire to occur. Following the same idea, fuel removal seems to be one of the best strategies to secure forest health. We also consider the effect of droughts, which are of high interest in recent times, which appear to lead to dramatic consequences, mainly increasing the risk to collapse.

%\paragraph{} % Opening, here on in discussion 
%The literature about early warning signal is quite extensive, and interdisciplinary. However, most of the methods used are based on the same mathematical phenomena, a bifurcation. For system, like the present one, who do not bifurcate to have substantial change, it appears that new methods need to be constructed.
 



%\subsection*{Synthesis}
%\addcontentsline{toc}{subsection}{Synthesis}


%\subsection*{Opening}    %%%%% Already in the discussions part !!!
%\addcontentsline{toc}{subsection}{Opening}

%\paragraph{}
%Talk about a more general / different problem ...



\newpage
\addcontentsline{toc}{section}{Bibliography}
%\bibliographystyle{plain}
%\bibliographystyle{alpha}
%\bibliographystyle{apalike}
\bibliographystyle{plainnat}

%\bibliography{references}
\bibliography{references_zotero,references}


%%%%%%%%%%%%%%%%%%%%%%%%%%%%%%%%%%%%%%%%%%%%%%%%%%%%%%%%%%%%%%%%%%%%%%%%%%%%%%%%%%%%%%%%%%%%%%%%%%%%%%
%%%%%%%%%%%%%%%%%%%%%%%%%%%%%%%%%%%%%%%%%%%%% Appendices %%%%%%%%%%%%%%%%%%%%%%%%%%%%%%%%%%%%%%%%%%%%%
%%%%%%%%%%%%%%%%%%%%%%%%%%%%%%%%%%%%%%%%%%%%%%%%%%%%%%%%%%%%%%%%%%%%%%%%%%%%%%%%%%%%%%%%%%%%%%%%%%%%%%

\newpage
\appendix
\addcontentsline{toc}{section}{Annexes}

\newpage
\section{Adimensionnalisation}
\label{adim}

\paragraph{}
The origianl model is the following:

\[
\left\lbrace
\begin{array}{rcl}
\frac{dn}{ds} & = & gN(1-\frac{n}{K})(\frac{n}{a}-1) - \delta_F(t)\phi(t)(n+\alpha w) \\
\\
\frac{dw}{ds} & = & \mu N - \eta W - \beta\delta_F(t)\phi(t)(N+\alpha W) \\
\end{array}
\right.
\]

\paragraph{}
We have two different dimension that can be removed: "biomass density" and time. We can remark that we could also consider that $N$ and $W$ have not the same dimension, and thus we could remove one more parameter. However, this would do not conserve anymore the respective proportion of $N$ and $W$.

\paragraph{}
Note
\[
\begin{array}{rcl}
N & = & \frac{n}{\lambda} \\
\\
W & = & \frac{w}{\lambda} \\
\\
t & = & \frac{s}{\tau} \\
\end{array}
\]

\paragraph{}
We have thus the following system,
\[
\begin{array}{rl}
& 

\left\lbrace
\begin{array}{rcl}
\frac{\lambda}{\tau}\frac{dN}{dt} & = & g\lambda N(1-\frac{\lambda N}{K})(\frac{\lambda N}{a}-1) - \delta_F(t)\phi(t)(\lambda N+\alpha \lambda W) \\
\\
\frac{\lambda}{\tau}\frac{dW}{dt} & = & \mu \lambda N -\eta \lambda W - \beta\delta_F(t)\phi(t)(\lambda N+\alpha \lambda W) \\
\end{array}
\right.
\\
\\
\Leftrightarrow & 
\left\lbrace
\begin{array}{rcl}
\frac{dN}{dt} & = & \frac{g \lambda}{\tau a} N(1-\frac{\lambda N}{K})(N-\frac{a}{\lambda}) - \delta_F(t)\tau\phi(t)(N+\alpha W) \\
\\
\frac{dW}{dt} & = & \mu \tau N -\eta \tau W - \beta\delta_F(t)\tau\phi(t)(N+\alpha W) \\
\end{array}
\right.
\end{array}
\]


If we take 
\[
\begin{array}{rcl}
\lambda & = & K \\
\tau & = & \frac{g\lambda}{a} \\
A & = & \frac{a}{K} \\
s(t) & = & \tau\phi(t) \\
m & = & \tau\mu \\
d & = & \tau\eta
\end{array}
\]
\paragraph{}
We have finally the following system,
\[
\left\lbrace
\begin{array}{rcl}
\frac{dN}{dt} & = & N(1-N)(N-A) - \delta_F(t)s(t)(N+\alpha W) \\
\frac{dW}{dt} & = & mN -dW - \beta\delta_F(t)s(t)(N+\alpha W) \\
\end{array}
\right.
\]



\newpage
\section{Simple results}



%\paragraph{Remark} % equilibrium
\paragraph{} % equilibrium
%\todo{COMMENTS: - Before the simple analysis, you say "we only consider growth terms.", but you also consider decay and such. You can say you look at the continuous dynamics without the effect of fires. Or something else similar.
%- I don't understand this sentence "Also, the figure below give an example for a well choose set of parameters of a time series.". What is chosen? You can instead write something like "We demonstrate the dynamics of this model in figure 2."
%- I was confused by the mysterious short paragraph "We can remark that the growth of the system is continuous and deterministic. However, the perturbation (fire) are discrete and stochastic. The implementation to solve the system need thus to take care of this, more details in appendix."
%- I am not sure what the purpose is of the end of page 14 and beginning of page 15. It does not seem to be necessary to understand other parts in the methods section, as far as I can tell. Maybe put this in the results section? Also, I don't know the value of the phase-portrait (fig.3) - you don't explain it in a legend, or really mention it in the main text. }


To help understand the dynamics of the model used, we give some simple analytic result. to do this, we only consider growth terms.

\[ % adim model
\left\lbrace
\begin{array}{rcl}
\frac{dN}{dt} & = & N(1-N)(N-A) \\
\\
\frac{dW}{dt} & = & mN -dW \\
\end{array}
\right.
\]


\paragraph{}
We will show that we have 3 equilibrium, 2 stable $(0, 0)$ and $(1, \frac{m}{d})$ and one unstable $(A, \frac{mA}{d})$.

\paragraph{}
Also, the figure below give an example for a well choose set of parameters of a time series. We can see that here the dynamics come back to the stable equilibrium $(1, \frac{m}{d})$ most of the time, except for the last fire when the system collapse to go to the other stable equilibrium $(0, 0)$. %\todo{strength the temporal series (final time = 300)}

\begin{figure}[h!]
\centering
\includegraphics[width=6cm]{return_between_2.png}
\caption{Time series example}
%\label{fig:universe}
\end{figure}


\paragraph{}
Also we represent the phase portrait for both $N$ and $W$. We recall that the dynamics of $W$ depend of the value $N$. 
%%%%% same parameters set for both

\begin{figure}[h!]
\centering
\includegraphics[width=6cm]{phase_N.png}
\includegraphics[width=6cm]{phase_W.png}
\caption{Phase portrait}
%\label{fig:universe}
\end{figure}

%\paragraph{}
%We can remark that the growth of the system is continuous and deterministic. However, the perturbation (fire) are discrete and stochastic. The implementation to solve the sytem need thus to take care of this, more details in \hyperref[technicality]{appendix}.
%\todo{I have already 2-3 papers about this, re found, cite}




%\subsection{Particular point}


\subsection{Equilibrium}
\label{equi}
\paragraph{}
We want to determine the equilibrium point for the deterministic part:
\[
\left\lbrace
\begin{array}{rcl}
\frac{dN}{dt} & = & N(1-N)(N-A) \\
\\
\frac{dW}{dt} & = & mN -dW \\
\end{array}
\right.
\]
This point are characterise by
\[
\left\lbrace
\begin{array}{rcl}
N(1-N)(N-A) & = & 0\\
mN -dW & = & 0\\
\end{array}
\right.
\]
To be an equilibrium point $N(1-N)(N-A)$ need to be equal to $0$. \\
Thus, $N = 0$, $N = 1$ or $N = A$. \\
For $N = 0$, we have $W = \frac{m}{d}N = 0$. \\
For $N = 1$, we have $W = \frac{m}{d}N = \frac{m}{d}$. \\
For $N = A$, we have $W = \frac{m}{d}N = \frac{m}{d}A$. \\

\paragraph{}
Finally, we have 3 equilibrium point,
\[
\begin{array}{c}
    \left\lbrace
    \begin{array}{rcl}
    N & = & 0 \\
    W & = & 0 \\
    \end{array}
    \right.
    \\
    \\
    \left\lbrace
    \begin{array}{rcl}
    N & = & 1 \\
    W & = & \frac{m}{d} \\
    \end{array}
    \right.
    \\
    \\
    \left\lbrace
    \begin{array}{rcl}
    N & = & A \\
    W & = & \frac{m}{d}A \\
    \end{array}
    \right.
\end{array}
\]

\subsection{Stability}

Here we want to study the stability of each equilibrium of the following system,

\[
\left\lbrace
\begin{array}{rcl}
\frac{dN}{dt} & = & N(1-N)(N-A) \\
\\
\frac{dW}{dt} & = & mN -dW \\
\end{array}
\right.
\]

For the point $(0,0)$, we linearize, by noting 
\[
\begin{array}{rcl}
\epsilon_N & = & N - 0 \\
\epsilon_W & = & W - 0 \\
\end{array}
\]
The system become
\[
\left\lbrace
\begin{array}{rcl}
\frac{d\epsilon_N}{dt} & = & \epsilon_N(1-\epsilon_N)(\epsilon_N-A) \\
\\
\frac{d\epsilon_W}{dt} & = & m\epsilon_N -d\epsilon_W \\
\end{array}
\right.
\]
For both $\epsilon_N$ and $\epsilon_W$ small, we have
\[
\left\lbrace
\begin{array}{rcl}
\frac{d\epsilon_N}{dt} & = & -\epsilon_N A \\
\\
\frac{d\epsilon_W}{dt} & = & m\epsilon_N -d\epsilon_W \\
\end{array}
\right.
\]
The eigenvalues of the Jacobian are, $-A$ and $-d$ because both $A$ and $d$ are positive, we have a stable point.
\\
\\
For the point $(A,\frac{m}{d}A)$, we linearize, by noting 
\[
\begin{array}{rcl}
\epsilon_N & = & N - A \\
\epsilon_W & = & W - \frac{m}{d}A \\
\end{array}
\]
The system become
\[
\left\lbrace
\begin{array}{rcl}
\frac{d\epsilon_N}{dt} & = & (A+\epsilon_N)(1-A-\epsilon_N)\epsilon_N \\
\\
\frac{d\epsilon_W}{dt} & = & m(\epsilon_N+A) -d(\epsilon_W+\frac{m}{d}A) \\
\end{array}
\right.
\]
For both $\epsilon_N$ and $\epsilon_W$ small, we have
\[
\left\lbrace
\begin{array}{rcl}
\frac{d\epsilon_N}{dt} & = & A(1-A)\epsilon_N \\
\\
\frac{d\epsilon_W}{dt} & = & m(\epsilon_N+A) -d(\epsilon_W+\frac{m}{d}A) \\
\end{array}
\right.
\]
An eigenvalue is $A(1-A)$, because $A\in(0,1)$ the eigenvalue is positive, and the point $(A,\frac{m}{d}A)$ is unstable.
\\
\\
For the point $(1,\frac{m}{d})$, we linearize, by noting 
\[
\begin{array}{rcl}
\epsilon_N & = & N - 1 \\
\epsilon_W & = & W - \frac{m}{d} \\
\end{array}
\]
The system become
\[
\left\lbrace
\begin{array}{rcl}
\frac{d\epsilon_N}{dt} & = & (A+\epsilon_N)(-\epsilon_N)(1-A+\epsilon_N) \\
\\
\frac{d\epsilon_W}{dt} & = & m(\epsilon_N-1) -d(\epsilon_W+\frac{m}{d}) \\
\end{array}
\right.
\]
For both $\epsilon_N$ and $\epsilon_W$ small, we have
\[
\left\lbrace
\begin{array}{rcl}
\frac{d\epsilon_N}{dt} & = & -(1-A)\epsilon_N \\
\\
\frac{d\epsilon_W}{dt} & = & m(\epsilon_N-1) -d(\epsilon_W+\frac{m}{d}) \\
\end{array}
\right.
\]
The eigenvalues of the Jacobian are, $(1-A)$ and $-m$ because $m$ is positive and $a\in(0,1)$, we have a stable point.


\paragraph{}
To conclude, we have an unstable point $(A, \frac{m}{d}A)$ and two stable point, the origin $(0,0)$ and $(1, \frac{m}{d})$.


\newpage
\section{Solve the system}

\label{technicality}

\paragraph{}
This appendices relate different problem faced to solve the dynamical system.


\subsection{Algorithm}

\paragraph{}
As explain previously, the main to solve this system is the stochastic events. Indeed, the usual method to solve dynamical system, typically Runge-Kutta \citep{butcher1964implicit} used for each step actual but also previous estimation of the derivatives to better approximate the solutions. However, the non smoothness of the solutions due to the fire disturbance, prevent the use of this traditional method.

The method choose here is to solve to stop the resolution of the system for each fire. In other word, we use classical method such as Runge-Kutta between each fire, and compute the effect of the fire when one occur and remove the correspond biomass.



\begin{algorithm}
\caption{Solver}
\begin{algorithmic}
\REQUIRE $Initial\_point$, $nbre\_iter$, $dt$
\ENSURE $Final point$
\STATE $Time = [0, dt, 2dt, ..., nbre\_iter \times dt]$
\STATE $c=0$
\WHILE{$c < nbre\_iter$}
    \IF{$Fire[c] == False$}
        \STATE $Sequence = [Time[c]]$
        \STATE  $c = c + 1$
        \WHILE{$c < nbre\_iter$ \AND $Fire[c] == False$}
            \STATE $Sequence += [Time[c]]$
            \STATE $c = c + 1$
        \ENDWHILE
        \STATE $Y = solve\_sequence(Initial\_point, Sequence)$
    \ELSE
        \STATE $initial\_point = Y - density\_burned(Y, c)$
        \STATE $initial\_point = max(initial\_point, 0)$
        \STATE $Y = solve\_sequence(initial\_point, [Time[c-1], Time[c]])$
        \STATE $c = c + 1$
    \ENDIF
\ENDWHILE
\end{algorithmic}
\end{algorithm}

\newpage

\subsection{Choice of the time step}

\paragraph{}
In the study, frequency of the fire is a major concern, as a consequence, the time step need to be adapted. Indeed, it is important to at least several time step between each fire, in order to simulate the growth of the biomass. However, as always, a too small time step increased the time needed to compute the solutions. Here we choose the following relation to assign the value of the time step:
\[
dt = \min(0.1, \frac{0.1}{frequency})
\]



\subsection{Time of the study}

\paragraph{}
Same as the time step, the choice of the final time of the study is a compromised between the time needed to resolve the system (linear to the final time) and the robustness of the numerical approximation. In particular, because the model used stochastic events it is important to simulate for a long enough time in order to have a correct approximations of the variability.
Also, we can remark that the final time change the value of the collapse probability (more we wait, more the system risk to collapse). However, the choice of final time do not affect the collapse probability by time units.

%\todo{plot evolution des measures VS time study, on doir voir une oscillation qui converge, pour var et cp by time units et une augmentation de cp}

\newpage
\section{Average estimation}
\label{average}

\paragraph{}
In order to not have to run too long simulation, an average of both $N$ and $W$ is estimated. It help to initialise the dynamics, indeed the transitions time is shorter
%\todo{ ? ? ? (plot time series in appendices, to compare if we take the initial point [1.,0] or $[1,\frac{m}{d}]$ explain that in this case we need to have long enough simulation especially if $d$ the recovery rate is too small)}

We use the following model:
\[
\left\lbrace
\begin{array}{rcl}
\frac{dN}{dt} & = & N(1-N)(N-A) - \delta_F(t)s(t)(N+\alpha W) \\
\frac{dW}{dt} & = & mN -dW - \beta\delta_F(t)s(t)(N+\alpha W) \\
\end{array}
\right.
\]
Assumption: frequency is high enough: 
\[
\begin{array}{rcl}
\delta_F(t) & \approx & f \\
s(t) & \approx & s \\
\end{array}
\]
With $f$ the average frequency of the fire and $s$ the average strength of the fire.


\[
\left\lbrace
\begin{array}{rcl}
\frac{dN}{dt} & = & N(1-N)(N-A) - f s (N+\alpha W) \\
\frac{dW}{dt} & = & mN -dW - \beta f s (N+\alpha W) \\
\end{array}
\right.
\]
At the pseudo equilibrium (when fire counterbalance growth).
\[
\left\lbrace
\begin{array}{rcl}
\frac{dN^*}{dt} & = & 0 \\
\frac{dW^*}{dt} & = & 0 \\
\end{array}
\right.
\]
Focus on the second equation: 
\[
\begin{array}{rcl}
\frac{dW^*}{dt} & = & 0 \\
mN^*-dW^* -\beta f s (N^*+\alpha W^*) & = & 0 \\
(m-\beta s f)N^* - (d+\beta f s \alpha) W^* & = & 0 \\
W^* & = & \frac{m-\beta s f}{d + \beta s f \alpha} N^*
\end{array}
\]
Notation
\[
\begin{array}{rcl}
\epsilon & = & \frac{m-\beta s f}{d + \beta s f \alpha} \\
\end{array}
\]
Thus
\[
W^* = \epsilon N^*
\]
For the first equation
\[
\begin{array}{rcl}
N^*(1-N^*)(N^*-A) - f s (N^*+\alpha W^*) & = & 0 \\
-AN^*+(1+A)N^{*^2}-N^{*^3} -f s (1+\alpha\epsilon)N^* & = & 0 \\
-(A+f s (1+\alpha\epsilon))N^*+(1+A)N^{*^2}-N^{*^3} & = & 0 \\
\end{array}
\]
Denote $\gamma = sf(1+\alpha\epsilon)$. 
\begin{equation}
-(A+\gamma)N^*+(1+A)N^{*^2}-N^{*^3} = 0
\end{equation}
The trivial solution $N^* = 0$ have no interest (in this case, both alive and death biomass are $0$).
\[
\begin{array}{rcl}
-(A+\gamma)+(1+A)N^{*}-N^{*^2} & = & 0 \\
\end{array}
\]

\[
\begin{array}{rcl}
\Delta & = & (1+A)^2 - 4(A+\gamma) \\
& = & (1-A)^2 - 4\gamma \\
\end{array}
\]
For now, we assume it is positive. %\todo{use it to delimit cases ?}
In practice, it is enough to have the product $f.s$ low enough %(here $f$ is high, but it is still possible to decrease $s$.
\[
\left\lbrace
\begin{array}{rcl}
N_1^* & = & \frac{1+a-\sqrt{(1-a)^2-4\gamma}}{2} \\
N_2^* & = & \frac{1+a+\sqrt{(1-a)^2-4\gamma}}{2}
\end{array}
\right.
\]
By using (1), because $\gamma$ and $A$ are both positives, we can deduce from the order of the solution $(0, N_1^*, N_2^*)$ that $N_1^*$ is unstable and $N_2^*$ is stable. 
\\
We are only interested in the stable solution, so, from now, we use the notation $N^* = N_2^*$.



\begin{figure}[h!]
\centering
\includegraphics[width=7cm]{average.png}
\caption{Estimation and measures of the average of $N$ and $W$}
%\label{fig:universe}
\end{figure}

\paragraph{Remark}
Sometimes, the approximation of $W^*$ take negative value. Because $W$ can not be negative, we only consider the positive part of $W^*$.




%\newpage
%\section{proba per time unit}
%\label{proba_per_time_unit}

%\todo{exact derivation}
%\todo{algorithm}
%\todo{plot to check / illustrate}


\newpage
\section{Variability algorithm}

%\subsection{algo variability}
\label{algo_variability}
%\todo{algo variability}



\begin{algorithm}
\caption{Variability}
\begin{algorithmic}
\REQUIRE $Time\_series\_for\_each\_simulations$
\ENSURE $variability\_approximation$
\STATE $c=0$
\FOR{$i=1$ \TO $simulation\_number$}
    \IF{$Time\_series[i]$ do not collapse before $t=0.2$ of the $final\_time$}
        \STATE $c = c + 1$
        \STATE $variability\_approximation$ = $variability\_approximation$ + variance(biomass between $10\%$ and $20\%$ of the final time study} 
    \ENDIF
\ENDFOR
\STATE $variability\_approximation = \frac{variability\_approximation}{c}$
\end{algorithmic}
\end{algorithm}






%\subsection{other variability}
%\label{other_variability}
%\todo{other variability, present them, their advantages, drawback, and their algorithm}


\newpage
\section{Variability estimation}
%\subsection{variability estimation derivation}
\label{variability_estimation}
%\todo{variability estimation}

%\subsection{variability estimation other}
%\label{variability_estimation_other}
%\todo{variability estimation other}


Here, we want to estimate the variability of $N$. For this, we use the following derivation.
Firstly, we still assume that $f$ is high and $s$ low. 
\\
We denote $U$ the growth rate of $N$ at the equilibrium $N$. \\
The growth of $N$ is driven by  
\[
\frac{dN}{dt} = N(1-N)(N-A)
\]
Which is simply the first equation of the model without the perturbation term.
We use the notation
\[
X = N-1
\]
Thus the previous equation becomes
\[
\begin{array}{rcl}
\frac{dX}{dt} & = & (X+1)X(X+1-A) \\
& = & 1.X(1-A) + o_0(X) \\
\end{array}
\]
Finally, we have a growth rate 
\[
U = 1-A
\]
Also we denote, $\lambda$ the average severity of a fire. The derivation of $\lambda$ is more complex and will be studied further.

%In order to estimate the variability, we assume that $f$ is high enough to make decompose the variability as the product of the frequency and the perturbations.
Because, the strength is low, the frequency is high, according to \cite{zelnik2018impact} we can estimate the variability by: 
\[
\begin{array}{rcl}
Variability & = & f\frac{\lambda^2}{U}
\end{array}
\]
Thus, the main difficulty to estimate the variability is to approximate $\lambda$. An important point is that when no fuel is present the fire stop. This implies that the actual average of the strength is lower than $s$. In order to consider this point, we calculate again the average of the perturbation. To have a fire we need that to have more fuel than the biomass burned:
\[
\begin{array}{rcl}
W^{av} & > & s\beta(N^{av}+\alpha W^{av}) \\
s & < & \frac{W^{av}}{\beta(N^{av}+\alpha W^{av})} \\
\end{array}
\]
In order to respect this threshold, we can simply redefine $\lambda$ as 
\[
\lambda = min(\{s(N^{av}+\alpha W^{av}), \frac{W^{av}}{\beta s (N^{av}+\alpha W^{av})}\})
\]





\newpage
\section{Collapse probability estimation}
\label{cp_derivation}
%\todo{}

\paragraph{}
We want to estimate $cp1$: the risk to collapse with only one fire. To do this we consider the exponential distribution of the strength, and integrate between $N^*-a$ and $\frac{W^*}{\beta}$. The first term is the quantity of biomass needed to go from the average $N^{av}$ to $A$ the Allee threshold effect, in order to collapse the system. And the second term is the maximum severity of the fire allowed by the fuel level (by assuming that it is closed to $W^{av}$).

\[
\begin{array}{rcl}
cp1 &=& \int_{N^*-a}^{\frac{W^*}{\beta}}s(x)(N^*+\alpha W^*)dx \\
&=& \int_{N^*-a}^{\frac{W^*}{\beta}} \frac{x}{s}\exp(-\frac{x}{s})(N^*+\alpha W^*)dx \\
&=& (N^*+\alpha W^*)([-x\exp(-\frac{x}{s})]_{N^*-a}^{\frac{W^*}{\beta}} +  [ -s\exp(-\frac{x}{s})]_{N^*-a}^{\frac{W^*}{\beta}}) \\
&=& (N^*+\alpha W^*)((N^*-a)\exp(-\frac{N^*-a}{s}) - \frac{W^*}{\beta}\exp(-\frac{W^*}{s\beta}) + s(\exp(-\frac{N^*-a}{s})- \exp(-\frac{W^*}{s\beta})) \\
cp1 &=& (N^*+\alpha W^*)((N^*-a+s)\exp(-\frac{N^*-a}{s}) - (\frac{W^*}{\beta}+s)\exp(-\frac{W^*}{s\beta}) \\
\end{array}
\]




%\newpage
%\section{Stability}
%\subsection{stability others}
%\label{stability_others}
%\todo{talk about others "measures of stability" such as skewness, kurtosis ... }


%\subsection{General notions of stability}
%\todo{Talk about the general concept of stability (across the different disciplines)}





\end{document}
